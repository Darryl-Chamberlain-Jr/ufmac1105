\documentclass{ximera}
\usepackage{sagetex}
%% handout
%% space
%% newpage
%% numbers
%% nooutcomes

%% You can put user macros here
%% However, you cannot make new environments

\graphicspath{{./}{module1Activity/}{module2Activity/}{module3Activity/}}

\usepackage{sagetex}
\usepackage{tikz}
\usepackage{hyperref}
\usepackage{tkz-euclide}
\usetkzobj{all}
\pgfplotsset{compat=1.7} % prevents compile error.

\tikzstyle geometryDiagrams=[ultra thick,color=blue!50!black]
 %% we can turn off input when making a master document

\outcome{Understand and solve linear inequalities.}
\author{Darryl Chamberlain Jr.}
 
\title{Objective 3 - Solve Compound Linear Inequalities}

\begin{document}
\begin{abstract}
Solve linear inequalities. 
\end{abstract}
\maketitle

\href{https://cnx.org/contents/mwjClAV_@8.1:uIjtHMfW@9/Linear-Inequalities-and-Absolute-Value-Inequalities}{Link to section in online textbook.}

%%%%%%%%%%%%%%%%%%%%%
%%%  Objective 2  %%%
%%%%%%%%%%%%%%%%%%%%%

First, watch the video below to learn how to solve compound inequalities. You can \href{}{use the notes here} to follow along with the video and record your thoughts.

\youtube{}

\begin{sagesilent}
import random
x = var('x')
##################
def maybeMakeNegative(rational):
    maybeNegative = (-1)**ZZ.random_element(2)
    rational = maybeNegative * rational
    return rational
    
def createNumerators():
    numerators = [0, 0, 0, 0]
    numerators[0] = maybeMakeNegative(ZZ.random_element(3, 11))
    numerators[1] = maybeMakeNegative(ZZ.random_element(3, 11))
    numerators[2] = maybeMakeNegative(ZZ.random_element(3, 11))
    numerators[3] = maybeMakeNegative(ZZ.random_element(3, 11))
    return numerators

def createDenominators():
    listOfDenominators= range(2, 10)
    a, b, c, d = random.sample(listOfDenominators, 4)
    return [Integer(a), Integer(b), Integer(c), Integer(d)]

def createCoefficientsHard():
    n0, n1, n2, n3 = createNumerators()
    d0, d1, d2, d3 = createDenominators()
    left = (n0/d0)+(n1/d1)*x
    right = (n2/d2)*x+(n3/d3)
    oneSolutionCheck = (n1/d1) - (n2/d2)
    while oneSolutionCheck == 0:
        n0, n1, n2, n3 = createNumerators()
        d0, d1, d2, d3 = createDenominators()
        left = (n0/d0)+(n1/d1)*x
        right = (n2/d2)*x+(n3/d3)
        oneSolutionCheck = (n1/d1) - (n2/d2)
    return [n0, n1, n2, n3, d0, d1, d2, d3]

def createIntervalSolutionHard(coefficients, direction):
    n0, n1, n2, n3, d0, d1, d2, d3 = coefficients
    left = (n0/d0)+(n1/d1)*x
    right = (n2/d2)*x+(n3/d3)
    # Checks which direction the interval points
    endpoint = round(float(solve(left-right, x)[0].rhs() ), 3)
    nearEndpoint = endpoint - 1
    checkNearby = float(float(n0/d0)+float(n1/d1)*nearEndpoint - (float(n2/d2)*nearEndpoint+float(n3/d3)))
    if direction == "leq" or direction == "less":
        if (checkNearby < 0):
            interval = [-oo, endpoint]
            whichSideFloat = "Right"
        else:
            interval = [endpoint, oo]
            whichSideFloat = "Left"  
    elif direction == "geq" or direction == "greater": 
        if (checkNearby > 0):
            interval = [-oo, endpoint]
            whichSideFloat = "Right"
        else:
            interval = [endpoint, oo]
            whichSideFloat = "Left"
    else:
        interval = [0, 0]
        whichSideFloat = "Neither"
    return [interval, whichSideFloat]

######## QUESTION 1 - Forces $(-\infty, a) \text{ or } (b, \infty)$ solution ##########
coefficients1A = createCoefficientsHard()
solution1A = createIntervalSolutionHard(coefficients1A, "less")
while solution1A[1] == "Left":
    coefficients1A = createCoefficientsHard()
    solution1A = createIntervalSolutionHard(coefficients1A, "less")
n1A0, n1A1, n1A2, n1A3, d1A0, d1A1, d1A2, d1A3 = coefficients1A
displayLeftFactor1A = n1A0/d1A0+n1A1/d1A1*x
displayRightFactor1A = n1A2/d1A2*x+n1A3/d1A3

coefficients1B = createCoefficientsHard()
solution1B = createIntervalSolutionHard(coefficients1B, "greater")
while solution1B[1] == "Right":
    coefficients1B = createCoefficientsHard()
    solution1B = createIntervalSolutionHard(coefficients1B, "greater")
n1B0, n1B1, n1B2, n1B3, d1B0, d1B1, d1B2, d1B3 = coefficients1B
displayLeftFactor1B = n1B0/d1B0+n1B1/d1B1*x
displayRightFactor1B = n1B2/d1B2*x+n1B3/d1B3


######## QUESTION 2 - Forces $(-\infty, a] \text{ or } [b, \infty)$ solution ##########
coefficients2A = createCoefficientsHard()
solution2A = createIntervalSolutionHard(coefficients2A, "leq")
while solution2A[1] == "Left":
    coefficients2A = createCoefficientsHard()
    solution2A = createIntervalSolutionHard(coefficients2A, "leq")
n2A0, n2A1, n2A2, n2A3, d2A0, d2A1, d2A2, d2A3 = coefficients2A
displayLeftFactor2A = n2A0/d2A0+n2A1/d2A1*x
displayRightFactor2A = n2A2/d2A2*x+n2A3/d2A3

coefficients2B = createCoefficientsHard()
solution2B = createIntervalSolutionHard(coefficients2B, "geq")
while solution2B[1] == "Right":
    coefficients2B = createCoefficientsHard()
    solution2B = createIntervalSolutionHard(coefficients2B, "geq")
n2B0, n2B1, n2B2, n2B3, d2B0, d2B1, d2B2, d2B3 = coefficients2B
displayLeftFactor2B = n2B0/d2B0+n2B1/d2B1*x
displayRightFactor2B = n2B2/d2B2*x+n2B3/d2B3

\end{sagesilent}

% Compound Inequality - OR
\begin{question}
$\sage{displayLeftFactor1A} < \sage{displayRightFactor1A}$ or $\sage{displayLeftFactor1B} > \sage{displayRightFactor1B}$

$\answer[format=string]{(} \answer{\sage{-Infinity}}, \answer[tolerance=0.05]{\sage{solution1A[0][1]}} \answer[format=string]{)}$
\\
$\cup$ 
\\
$\answer[format=string]{(} \answer[tolerance=0.05]{\sage{solution1B[0][0]}}, \answer{\sage{Infinity}} \answer[format=string]{)}$

\begin{hint}
There are four boxes so you can input the entire interval. Each interval should be: \\
( or [ \\
number or $\infty$ \\
number or $\infty$ \\
) or ] 
\end{hint}

\end{question}

% Compound Inequality - OR
\begin{question}
$\sage{displayLeftFactor2A} \leq \sage{displayRightFactor2A}$ or $\sage{displayLeftFactor2B} \geq \sage{displayRightFactor2B}$

$\answer[format=string]{(} \answer{\sage{-Infinity}}, \answer[tolerance=0.05]{\sage{solution2A[0][1]}} \answer[format=string]{]}$
\\
$\cup$ 
\\
$\answer[format=string]{[} \answer[tolerance=0.05]{\sage{solution2B[0][0]}}, \answer{\sage{Infinity}} \answer[format=string]{)}$

\begin{hint}
	There are four boxes so you can input the entire interval. Each option should be: \\
	( or [ \\
	number or $\infty$ \\
	number or $\infty$ \\
	) or ] 
\end{hint}
\end{question}

\begin{sagesilent}
x = var('x')
############
def createAllCoefficientsAndEndpoints():
    coefficients = [0, 0, 0, 0, 0, 0, 0]
    coefficients[0] = maybeMakeNegative(ZZ.random_element(3, 9))
    coefficients[1] = maybeMakeNegative(ZZ.random_element(3, 9))
    coefficients[3] = maybeMakeNegative(ZZ.random_element(3, 9))
    coefficients[4] = ZZ.random_element(3, 9)
    coefficients[5] = maybeMakeNegative(ZZ.random_element(3, 9))
    coefficients[6] = maybeMakeNegative(ZZ.random_element(3, 9))
    # Need 1, 4, and 6 set before 2
    coefficients[2] = max(coefficients[1], coefficients[6])*coefficients[4] + ZZ.random_element(2, 5)
    # This flips the inequalities
    smallerEndpoint = float((-coefficients[0]*coefficients[4] + coefficients[3]) / (coefficients[1]*coefficients[4] - coefficients[2]))
    largerEndpoint = float((coefficients[4]*coefficients[5] - coefficients[3]) / (coefficients[2] - coefficients[4] * coefficients[6]))
    #####
    # Makes sure we get a solution interval
    while  (largerEndpoint < smallerEndpoint) or (largerEndpoint == smallerEndpoint):
        coefficients[0] = maybeMakeNegative(ZZ.random_element(3, 9))
        coefficients[1] = maybeMakeNegative(ZZ.random_element(3, 9))
        coefficients[3] = maybeMakeNegative(ZZ.random_element(3, 9))
        coefficients[4] = ZZ.random_element(3, 9)
        coefficients[5] = maybeMakeNegative(ZZ.random_element(3, 9))
        coefficients[6] = maybeMakeNegative(ZZ.random_element(3, 9))
        # Need 1, 4, and 6 set before 2
        coefficients[2] = max(coefficients[1], coefficients[6])*coefficients[4] + ZZ.random_element(2, 5)
        # This flips the inequalities
        smallerEndpoint = float((-coefficients[0]*coefficients[4] + coefficients[3]) / (coefficients[1]*coefficients[4] - coefficients[2]))
        largerEndpoint = float((coefficients[4]*coefficients[5] - coefficients[3]) / (coefficients[2] - coefficients[4] * coefficients[6]))
    #####
    a, b, c, d, e, f, g = coefficients
    return [coefficients, smallerEndpoint, largerEndpoint]

def constructInequalitiesToDisplay(coefficients):
    c0, c1, c2, c3, c4, c5, c6 = coefficients
    andInequalityLeft = c0 + c1*x
    andInequalityMiddle = (c2*x + c3)/c4
    andInequalityRight = c5 + c6*x
    return [andInequalityLeft, andInequalityMiddle, andInequalityRight]
##############
coefficients3, smallerEndpoint3, largerEndpoint3 = createAllCoefficientsAndEndpoints()
andInequalityLeft3, andInequalityMiddle3, andInequalityRight3 = constructInequalitiesToDisplay(coefficients3)

coefficients4, smallerEndpoint4, largerEndpoint4 = createAllCoefficientsAndEndpoints()
andInequalityLeft4, andInequalityMiddle4, andInequalityRight4 = constructInequalitiesToDisplay(coefficients4)
\end{sagesilent}

% Compound Inequality - AND
\begin{question}
$$\sage{andInequalityLeft3} < \sage{andInequalityMiddle3} < \sage{andInequalityRight3}$$

$\answer[format=string]{(} \answer[tolerance=0.05]{\sage{smallerEndpoint3}}, \answer[tolerance=0.05]{\sage{largerEndpoint3}} \answer[format=string]{)}$

\begin{hint}
	There are four boxes so you can input the entire interval. Each option should be: \\
	( or [ \\
	number or $\infty$ \\
	number or $\infty$ \\
	) or ] 
\end{hint}
\end{question}

% Compound Inequality - AND
\begin{question}
$$\sage{andInequalityLeft4} \leq \sage{andInequalityMiddle4} \leq \sage{andInequalityRight4}$$

$\answer[format=string]{[} \answer[tolerance=0.05]{\sage{smallerEndpoint4}}, \answer[tolerance=0.05]{\sage{largerEndpoint4}} \answer[format=string]{]}$

\begin{hint}
	There are four boxes so you can input the entire interval. Each option should be: \\
	( or [ \\
	number or $\infty$ \\
	number or $\infty$ \\
	) or ] 
\end{hint}
\end{question}

\end{document}
