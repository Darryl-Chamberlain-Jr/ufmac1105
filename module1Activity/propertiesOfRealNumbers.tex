\documentclass{ximera}
%\usepackage{sagetex}
%% handout
%% space
%% newpage
%% numbers
%% nooutcomes

%%% You can put user macros here
%% However, you cannot make new environments

\graphicspath{{./}{module1Activity/}{module2Activity/}{module3Activity/}}

\usepackage{sagetex}
\usepackage{tikz}
\usepackage{hyperref}
\usepackage{tkz-euclide}
\usetkzobj{all}
\pgfplotsset{compat=1.7} % prevents compile error.

\tikzstyle geometryDiagrams=[ultra thick,color=blue!50!black]
 %% we can turn off input when making a master document

\outcome{Understand the different sets of numbers along with the properties of these sets.}
\author{Darryl Chamberlain Jr.}

\title{Order of Operations}

\begin{document}
\begin{abstract}
Apply the properties of Real numbers to simplify expressions.
\end{abstract}
\maketitle

\href{https://cnx.org/contents/mwjClAV_@8.1:0KhpF2RH@23/Real-Numbers-Algebra-Essentials}{Link to section in textbook}

%%%%%%%%%%%%%%%%%%%%%
%%%  Objective 3  %%%
%%%%%%%%%%%%%%%%%%%%%

Now that we have the terminology for the different sets of numbers, we can review their properties. We'll start with the Real numbers first. Watch \href{https://mediasite.video.ufl.edu/Mediasite/Play/adc9e0878901450db891a5b46b383bf01d}{this video} to review the properties of Real numbers. \textit{Note: You won't be asked to define a property or know the property by name. However, you \textbf{will} need to know how to use the properties to simplify in order.}

We'll focus on Order of Operations here as many students were taught an order that \textbf{does not align with how most calculators/computers simplify expressions.} Think about PEMDAS as:

\begin{question}
P: $\answer[format=string]{Parentheses}$

E: $\answer[format=string]{Exponents}$

M or D: $\answer[format=string]{Multiplication}$ or $\answer[format=string]{Division}$

A or S: $\answer[format=string]{Addition}$ or $\answer[format=string]{Subtraction}$
\end{question}

{\Large \textbf{If they are on the same level, you complete them from left-to-right.}}

\begin{question}
Let's take a closer look at why M/D is written on the same level.
$7 \div 5 \times 4 = \answer{5.6}$

$7 \times \frac{1}{5} \times 4 = \answer{5.6}$

Multiplying by $\frac{1}{5}$ is the same as dividing by $\answer{5}$. Now let's see what happens if we did multiplication first.

$7 \div (5 \times 4) = \answer{0.35}$

By changing everything to multiplication, we can see why it is so important to read from left-to-right when operations are on the same level!
\end{question}

Now try to simplify the more complicated expressions below.

\begin{sagesilent}
# Order of Operations Question
def generateSolution1(constants):
    c0, c1, c2, c3, c4, c5 = constants
    solution = float((c0 - ((c1/c2) * c3)) - (c4 * c5 ))
    return solution

def generateDistractor1(constants):
    c0, c1, c2, c3, c4, c5 = constants
    distractor = float(c0 - (c1/(c2 * c3)) - (c4 * c5 ))
    return distractor

def generateSolution2(constants):
    c0, c1, c2, c3, c4, c5 = constants
    solution = float(c0 - c1 - c2 + c3 - c4 + c5 )
    return solution

def generateDistractor2(constants):
    c0, c1, c2, c3, c4, c5 = constants
    distractor = float(c0 - c1 - (c2 + c3) - (c4 + c5) )
    return distractor

def generateSolution3(constants):
    c0, c1, c2, c3, c4, c5 = constants
    solution = float((c0/c1) - c2 + (c3/c4) + c5 )
    return solution

def generateDistractor3(constants):
    c0, c1, c2, c3, c4, c5 = constants
    distractor = float( (c0/c1) - ( (c2 + (c3/c4) ) + c5) )
    return distractor

def generateSolutionAndDistractor(structureType, constants):
    if structureType==1:
        return [generateSolution1(constants), generateDistractor1(constants)]
    elif structureType==2:
        return [generateSolution2(constants), generateDistractor2(constants)]
    elif structureType==3:
        return [generateSolution3(constants), generateDistractor3(constants)]
    else:
        return [0, 0]

def createProblem(structureType):
    # Array of 6 distinct integers
    constants = sample(list(range(2, 21)), 6)
    solution, distractor = generateSolutionAndDistractor(structureType, constants)
    # CHECKS if doing the question wrong will still give the correct solution.
    index = 0
    while solution == distractor:
        constants = sample(list(range(2, 21)), 6)
        solution, distractor = generateSolutionAndDistractor(structureType, constants)
        # Makes sure we don't get stuck in an infinite loop
        index += 1
        if (index > 100):
            break
    return [constants, solution, distractor]

########## END OF DEFINITIONS ###########

constants1, solution1, distractor1 = createProblem(1)
constants2, solution2, distractor2 = createProblem(2)
constants3, solution3, distractor3 = createProblem(3)

\end{sagesilent}

\begin{exercise}
Simplify the expression $\sage{constants1[0]} - \sage{constants1[1]} \div \sage{constants1[2]} * \sage{constants1[3]} - (\sage{constants1[4]} * \sage{constants1[5]})$.

$\answer[tolerance=0.01]{\sage{solution1}}$

\begin{hint}
Did you get $\sage{distractor1}$ as your answer? If you did, you are not treating multiplication and division as being on the same level. You need to complete these from left-to-right!
\end{hint}
\end{exercise}

\begin{exercise}
	Simplify the expression $\sage{constants2[0]} - \sage{constants2[1]} - \sage{constants2[2]} + \sage{constants2[3]} - \sage{constants2[4]} + \sage{constants2[5]}$.

	$\answer[tolerance=0.01]{\sage{solution2}}$
  \begin{hint}
  Did you get $\sage{distractor2}$ as your answer? If you did, you are not treating addition and subtraction as being on the same level. You need to complete these from left-to-right!
  \end{hint}
\end{exercise}

\begin{exercise}
	Simplify the expression $\sage{constants3[0]} \div \sage{constants3[1]} - \sage{constants3[2]} + \sage{constants3[3]} \div \sage{constants3[4]} + \sage{constants3[5]}$.

	$\answer[tolerance=0.01]{\sage{solution3}}$
  \begin{hint}
  Did you get $\sage{distractor3}$ as your answer? If you did, you are not treating addition and subtraction as being on the same level. You need to complete these from left-to-right!
  \end{hint}
\end{exercise}

\end{document}
