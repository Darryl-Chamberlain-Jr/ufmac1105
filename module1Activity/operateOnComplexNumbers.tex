\documentclass{ximera}
\outcome{Understand the different sets of numbers along with the properties of these sets.}
\author{Darryl Chamberlain Jr.}
\title{Operate on Complex Numbers}

\begin{document}
\begin{abstract}
Add, Subtract, Multiply, or Divide Complex numbers.
\end{abstract}
\maketitle

\href{https://cnx.org/contents/mwjClAV_@8.1:Sqk1HAGf@9/Complex-Numbers}{Link to section in textbook}

%%%%%%%%%%%%%%%%%%%%%
%%%  Objective 4  %%%
%%%%%%%%%%%%%%%%%%%%%

We end this lesson by looking at properties of the Complex numbers. Watch the video below to review the properties of Complex numbers. You can print out \href{http://people.clas.ufl.edu/dchamberlain31/files/Objective-4-Operate-on-Complex-Numbers.pdf}{these notes} to follow along and keep notes to organize your thoughts.

\youtube{eXZxMRyY5zc}

Adding and subtracting Complex numbers requires you to add/subtract like terms. As students are proficient with combing like terms, the homework will focus on multiplying and dividing Complex numbers. For each, simplify the expression into the form $a+bi$.

\begin{sagesilent}
import random
import cmath
from sympy import *
#
# Takes in a rational value and multiplies it by either 1 or -1.
def maybeMakeNegative(rational):
    maybeNegative = int((-1)**random.randint(0, 1))
    rational = maybeNegative * rational
    return rational
# Module 1 - Real & Complex Numbers
# Objective - Multiply two complex numbers.

### DEFINITIONS ###
def generateProblemCoefficients():
    listIntegers = range(2, 11)
    constants = random.sample(listIntegers, 4)
    constants = [maybeMakeNegative(i) for i in constants]
    a1, b1, a2, b2 = constants
    while (a1*b2 + b1*a2) == 0:
        constants = random.sample(listIntegers, 4)
        constants = [maybeMakeNegative(i) for i in constants]
        a1, b1, a2, b2 = constants
    return constants
def generateSolutionAndDistractors(coefficients):
    a1, b1, a2, b2 = coefficients
    complex1 = complex(a1, b1)
    complex2 = complex(a2, b2)
    product = complex1*complex2
    solution = [int(product.real), int(product.imag)]
    distractor1Product = complex(a1, -b1)*complex(a2, b2)
    distractor1 = [int(distractor1Product.real), int(distractor1Product.imag)]
    distractor2Product = complex(a1, b1)*complex(a2, -b2)
    distractor2 = [int(distractor2Product.real), int(distractor2Product.imag)]
    distractor3Product = complex(a1, -b1)*complex(a2, -b2)
    distractor3 = [int(distractor3Product.real), int(distractor3Product.imag)]
    distractor4Product = complex(a1*a2, b1*b2)
    distractor4 = [int(distractor4Product.real), int(distractor4Product.imag)]
    return [solution, distractor1, distractor2, distractor3, distractor4]

### VARIABLE DECLARATIONS ###
problemCoefficients = generateProblemCoefficients()
solutionList = generateSolutionAndDistractors(problemCoefficients)

### DEFINE XRONOS VARIABLES ###
xronosSolution = solutionList[0]
xronosDistractors = [solutionList[1], solutionList[2], solutionList[3], solutionList[4]]
displayXronosStem = "Simplify the expression below into the form $a+bi$."
xronosHint="You can treat $i$ as a variable and distribute. Just remember that $i^2=-1$, so you can continue to reduce after you distribute."
\end{sagesilent}

\begin{exercise}

\sage{displayXronosStem}

$\answer{\sage{xronosSolution[0]}} + \answer{\sage{xronosSolution[1]}} \, i$

\begin{hint}
\sage{xronosHint}
\end{hint}
\end{exercise}

\end{document}

% CREATE SECOND COPY IF THIS WORKS

\begin{sagesilent}
def maybeMakeNegative(rational):
    maybeNegative = (-1)**ZZ.random_element(2)
    rational = maybeNegative * rational
    return rational

def generateProblem():
    listIntegers = range(2, 11)
    c0, c1, c2, c3 = sample(listIntegers, 4)
    constants = [Integer(maybeMakeNegative(c0)), Integer(c1), Integer(maybeMakeNegative(c2)), Integer(-c3)]
    return constants

def generateSolution(complex1, complex2):
    quotient = complex1/complex2
    realPart = quotient.real()
    imagPart = quotient.imag()
    #quotient = CC(realPart, imagPart)
    return [realPart, imagPart]

######### END OF DEFINITIONS #############

######## QUESTION 12 ###########
a112, b112, a212, b212 = generateProblem()
complex12A = CC(a112, b112)
complex12B = CC(a212, b212)
answer12A, answer12B = generateSolution(complex12A, complex12B)

######## QUESTION 13 ###########
a113, b113, a213, b213 = generateProblem()
complex13A = CC(a113, b113)
complex13B = CC(a213, b213)
answer13A, answer13B = generateSolution(complex13A, complex13B)
\end{sagesilent}

\begin{exercise}
	$\frac{\sage{a112}+\sage{b112}i}{\sage{a212}-\sage{-b212}i}$

\begin{hint}
The goal when dividing by a Complex number is to remove the Complex number from the denominator. Is there a word for the number we can multiply by to remove the Complex part of a number?
\end{hint}

	$\answer[tolerance=0.01]{\sage{answer12A}}$ $+ \answer[tolerance=0.01]{\sage{answer12B}}i$
\end{exercise}

\begin{exercise}
	$\frac{\sage{a113}+\sage{b113}i}{\sage{a213}-\sage{-b213}i}$

	$\answer[tolerance=0.01]{\sage{answer13A}}$ $+ \answer[tolerance=0.01]{\sage{answer13B}}i$
\end{exercise}

\end{document}
