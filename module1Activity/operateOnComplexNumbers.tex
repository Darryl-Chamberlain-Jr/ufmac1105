\documentclass{ximera}
\outcome{Understand the different sets of numbers along with the properties of these sets.}
\author{Darryl Chamberlain Jr.}
\title{Operate on Complex Numbers}

\begin{document}
\begin{abstract}
Add, Subtract, Multiply, or Divide Complex numbers.
\end{abstract}
\maketitle

\href{https://cnx.org/contents/mwjClAV_@8.1:Sqk1HAGf@9/Complex-Numbers}{Link to section in textbook}

%%%%%%%%%%%%%%%%%%%%%
%%%  Objective 4  %%%
%%%%%%%%%%%%%%%%%%%%%

We end this lesson by looking at properties of the Complex numbers. Watch the video below to review the properties of Complex numbers. You can print out \href{http://people.clas.ufl.edu/dchamberlain31/files/Objective-4-Operate-on-Complex-Numbers.pdf}{these notes} to follow along and keep notes to organize your thoughts.

\youtube{eXZxMRyY5zc}

Adding and subtracting Complex numbers requires you to add/subtract like terms. As students are proficient with combing like terms, the homework will focus on multiplying and dividing Complex numbers. For each, simplify the expression into the form $a+bi$.

\begin{sagesilent}
def maybeMakeNegative(rational):
    maybeNegative = (-1)**ZZ.random_element(2)
    rational = maybeNegative * rational
    return rational

def generateProblem():
    listIntegers = range(2, 11)
    c0, c1, c2, c3 = sample(listIntegers, 4)
    constants = [Integer(maybeMakeNegative(c0)), Integer(c1), Integer(maybeMakeNegative(c2)), Integer(-c3)]
    return constants

def generateSolution(complex1, complex2):
    product = complex1 * complex2
    realPart = round(product.real(), 0)
    imagPart = round(product.imag(), 0)
    product = CC(realPart, imagPart)
    return product

############ END OF DEFINITIONS ##############

######### QUESTION 10 ###########
problem10 = generateProblem()
a110, b110, a210, b210 = problem10
complex10A = CC(a110, b110)
complex10B = CC(a210, b210)
answer10 = generateSolution(complex10A, complex10B)

######### QUESTION 11 ###########
problem11 = generateProblem()
a111, b111, a211, b211 = problem11
complex11A = CC(a111, b111)
complex11B = CC(a211, b211)
answer11 = generateSolution(complex11A, complex11B)
\end{sagesilent}

\begin{exercise}
$(\sage{a110}+\sage{b110}i)(\sage{a210}-\sage{-b210}i)$

$\answer{\sage{answer10}}$

\begin{hint}
Make sure you distribute and reduce $i^2$.
\end{hint}
\end{exercise}

\begin{exercise}
	$(\sage{a111}+\sage{b111}i)(\sage{a211}-\sage{-b211}i)$

	$\answer{\sage{answer11}}$
\end{exercise}

\begin{sagesilent}
def maybeMakeNegative(rational):
    maybeNegative = (-1)**ZZ.random_element(2)
    rational = maybeNegative * rational
    return rational

def generateProblem():
    listIntegers = range(2, 11)
    c0, c1, c2, c3 = sample(listIntegers, 4)
    constants = [Integer(maybeMakeNegative(c0)), Integer(c1), Integer(maybeMakeNegative(c2)), Integer(-c3)]
    return constants

def generateSolution(complex1, complex2):
    quotient = complex1/complex2
    realPart = quotient.real()
    imagPart = quotient.imag()
    #quotient = CC(realPart, imagPart)
    return [realPart, imagPart]

######### END OF DEFINITIONS #############

######## QUESTION 12 ###########
a112, b112, a212, b212 = generateProblem()
complex12A = CC(a112, b112)
complex12B = CC(a212, b212)
answer12A, answer12B = generateSolution(complex12A, complex12B)

######## QUESTION 13 ###########
a113, b113, a213, b213 = generateProblem()
complex13A = CC(a113, b113)
complex13B = CC(a213, b213)
answer13A, answer13B = generateSolution(complex13A, complex13B)
\end{sagesilent}

\begin{exercise}
	$\frac{\sage{a112}+\sage{b112}i}{\sage{a212}-\sage{-b212}i}$

\begin{hint}
The goal when dividing by a Complex number is to remove the Complex number from the denominator. Is there a word for the number we can multiply by to remove the Complex part of a number?
\end{hint}

	$\answer[tolerance=0.01]{\sage{answer12A}}$ $+ \answer[tolerance=0.01]{\sage{answer12B}}i$
\end{exercise}

\begin{exercise}
	$\frac{\sage{a113}+\sage{b113}i}{\sage{a213}-\sage{-b213}i}$

	$\answer[tolerance=0.01]{\sage{answer13A}}$ $+ \answer[tolerance=0.01]{\sage{answer13B}}i$
\end{exercise}

\end{document}
