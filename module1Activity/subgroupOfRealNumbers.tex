\documentclass{ximera}
%\usepackage{sagetex}
%% handout
%% space
%% newpage
%% numbers
%% nooutcomes

%%% You can put user macros here
%% However, you cannot make new environments

\graphicspath{{./}{module1Activity/}{module2Activity/}{module3Activity/}}

\usepackage{sagetex}
\usepackage{tikz}
\usepackage{hyperref}
\usepackage{tkz-euclide}
\usetkzobj{all}
\pgfplotsset{compat=1.7} % prevents compile error.

\tikzstyle geometryDiagrams=[ultra thick,color=blue!50!black]
 %% we can turn off input when making a master document

\outcome{Understand the different sets of numbers along with the properties of these sets.}
\author{Darryl Chamberlain Jr.}

\title{Subgroups of Real Numbers}

\begin{document}
\begin{abstract}
Identify the subgroup of Real numbers a number belongs to.
\end{abstract}
\maketitle

\href{https://cnx.org/contents/mwjClAV_@8.1:0KhpF2RH@23/Real-Numbers-Algebra-Essentials}{Link to section in online textbook}

%%%%%%%%%%%%%%%%%%%%%
%%%  Objective 1  %%%
%%%%%%%%%%%%%%%%%%%%%
First, watch the video below to review the different sets of Real numbers. You can print out \href{http://people.clas.ufl.edu/dchamberlain31/files/Objective-1-Subgroups-of-Real-Numbers.pdf}{these notes} to follow along and keep notes to organize your thoughts.

\youtube{RXXZOZBvYQs}

After watching the video, write down your own definitions for the following subgroups of the Real numbers. You should include examples for each (you may even want to take a sneak peak at the problems and use some of these as examples!) and descriptions of how to tell what the smallest set the number belongs to.

\begin{itemize}
\item Natural:
\item Whole:
\item Integers:
\item Rational:
\item Irrational:
\item Real:
\end{itemize}

We will test these definitions by categorizing the same set of numbers.

\begin{exercise}
Choose all of the following numbers that are \textbf{NATURAL} numbers.
  \begin{selectAll}
    \choice{$-\frac{21}{7}$}
    \choice{$-\frac{7}{21}$}
    \choice[correct]{$\frac{21}{7}$}
    \choice {$\frac{\pi}{4}$}
    \choice {$\frac{4}{\pi}$}
    \choice {$\frac{0}{\pi}$}
    \choice {$\frac{\pi}{0}$}
    \choice[correct] {$\sqrt{4}$}
    \choice {$\sqrt{-4}$}
    \choice {$\sqrt{21}$}
    \choice {$\sqrt{-21}$}
  \end{selectAll}

\begin{hint}
	Remember to reduce *first*, then think about what groups the number belongs to.
\end{hint}
\end{exercise}

\begin{exercise}
Choose all of the following numbers that are \textbf{Whole} numbers.
  \begin{selectAll}
    \choice {$-\frac{21}{7}$}
    \choice {$-\frac{7}{21}$}
    \choice[correct] {$\frac{21}{7}$}
    \choice {$\frac{\pi}{4}$}
    \choice {$\frac{4}{\pi}$}
    \choice[correct] {$\frac{0}{\pi}$}
    \choice {$\frac{\pi}{0}$}
    \choice[correct] {$\sqrt{4}$}
    \choice {$\sqrt{-4}$}
    \choice {$\sqrt{21}$}
    \choice {$\sqrt{-21}$}
  \end{selectAll}

\begin{hint}
	What is the only number included in the Whole numbers that is \textbf{not} included in the Natural numbers?
\end{hint}
\end{exercise}

\begin{exercise}
Choose all of the following numbers that are \textbf{Integer} numbers.
  \begin{selectAll}
    \choice[correct] {$-\frac{21}{7}$}
    \choice {$-\frac{7}{21}$}
    \choice[correct] {$\frac{21}{7}$}
    \choice {$\frac{\pi}{4}$}
    \choice {$\frac{4}{\pi}$}
    \choice[correct] {$\frac{0}{\pi}$}
    \choice {$\frac{\pi}{0}$}
    \choice[correct] {$\sqrt{4}$}
    \choice {$\sqrt{-4}$}
    \choice {$\sqrt{21}$}
    \choice {$\sqrt{-21}$}
  \end{selectAll}

\begin{hint}
	What is the biggest difference between Whole numbers and Integers?
\end{hint}
\end{exercise}

\begin{exercise}
Choose all of the following numbers that are \textbf{Rational} numbers.
  \begin{selectAll}
    \choice[correct] {$-\frac{21}{7}$}
    \choice[correct] {$-\frac{7}{21}$}
    \choice[correct] {$\frac{21}{7}$}
    \choice {$\frac{\pi}{4}$}
    \choice {$\frac{4}{\pi}$}
    \choice[correct] {$\frac{0}{\pi}$}
    \choice {$\frac{\pi}{0}$}
    \choice[correct] {$\sqrt{4}$}
    \choice {$\sqrt{-4}$}
    \choice {$\sqrt{21}$}
    \choice {$\sqrt{-21}$}
  \end{selectAll}

\begin{hint}
	What is the biggest difference between Integers and Rational numbers?
\end{hint}
\end{exercise}

\begin{exercise}
Choose all of the following numbers that are \textbf{Irrational} numbers.
  \begin{selectAll}
    \choice {$-\frac{21}{7}$}
    \choice {$-\frac{7}{21}$}
    \choice {$\frac{21}{7}$}
    \choice[correct] {$\frac{\pi}{4}$}
    \choice[correct] {$\frac{4}{\pi}$}
    \choice {$\frac{0}{\pi}$}
    \choice {$\frac{\pi}{0}$}
    \choice {$\sqrt{4}$}
    \choice {$\sqrt{-4}$}
    \choice[correct] {$\sqrt{21}$}
    \choice {$\sqrt{-21}$}
  \end{selectAll}

\begin{hint}
	What is the biggest difference between Rational and Irrational numbers?
\end{hint}
\end{exercise}

\begin{exercise}
Choose all of the following numbers that are \textbf{Real} numbers.
  \begin{selectAll}
    \choice[correct] {$-\frac{21}{7}$}
    \choice[correct] {$-\frac{7}{21}$}
    \choice[correct] {$\frac{21}{7}$}
    \choice[correct] {$\frac{\pi}{4}$}
    \choice[correct] {$\frac{4}{\pi}$}
    \choice[correct] {$\frac{0}{\pi}$}
    \choice {$\frac{\pi}{0}$}
    \choice[correct] {$\sqrt{4}$}
    \choice {$\sqrt{-4}$}
    \choice[correct] {$\sqrt{21}$}
    \choice {$\sqrt{-21}$}
  \end{selectAll}

\begin{hint}
	There are two ways to not be a Real number (that we know so far)...
\end{hint}
\end{exercise}

As you can see, there is a lot of overlap between these groups. You should also try to draw a picture like the video to represent how these subgroups interact. \textit{Since numbers belong to more than one group, the best way to describe these numbers is to identify the \textbf{smallest} subgroup they belong to.} The pictorial representation will help with this! \textbf{\Large Remember to reduce first, then decide the smallest subgroup the number belongs to!}

\begin{sagesilent}
def generateNatural():
	denominator = ZZ.random_element(15)+1
	numerator = denominator*(ZZ.random_element(15)+1)
	negOrNo = ""
	displayProblem = [numerator**2, denominator**2, negOrNo]
	return displayProblem
def generateWhole():
	denomBefore = ZZ.random_element(15)+1
	numerator = 0
	negOrNo = ""
	maybeComplexDisplayed = sample([1, 0], 1)
	if maybeComplexDisplayed == 1:
        denominator = latex(denomBefore*pi)
	else:
        denominator = denomBefore
	displayProblem = [numerator**2, denominator**2, negOrNo]
	return displayProblem
def generateInteger():
    denominator = ZZ.random_element(15)+1
	numerator = denominator*(ZZ.random_element(15)+1)
	negOrNo = "-"
	displayProblem = [numerator**2, denominator**2, negOrNo]
	return displayProblem
def generateRational():
	numeratorBefore = ZZ.random_element(15)+7
	denominatorBefore = ZZ.random_element(15)+7
	negative = ZZ.random_element(2)
	if negative == 0:
	    negOrNo = ""
	else:
	    negOrNo = "-"
	counter = 0
	while gcd(numeratorBefore, denominatorBefore) >= min(numeratorBefore, denominatorBefore):
	    numerator = ZZ.random_element(15)+7
	    denominator = ZZ.random_element(15)+7
	    counter = counter + 1
	numerator = numeratorBefore**2
	denominator = denominatorBefore**2
	displayProblem = [numerator, denominator, negOrNo]
	return displayProblem
def generateIrrational():
	numeratorBefore = ZZ.random_element(15)+1
	denominatorBefore = ZZ.random_element(15)+1
	negative = ZZ.random_element(2)
	if negative == 0:
	    negOrNo = ""
	else:
	    negOrNo = "-"
	integerType = type(sqrt(4))
	while type(sqrt(numeratorBefore/denominatorBefore)) == integerType:
	    numeratorBefore = ZZ.random_element(15)+1
	    denominatorBefore = ZZ.random_element(15)+1
	numerator = numeratorBefore
	denominator = denominatorBefore
	displayProblem = [numerator, denominator, negOrNo]
	return displayProblem
def generateNotReal():
	numeratorBefore = ZZ.random_element(15)+1
	denominatorBefore = ZZ.random_element(15)+1
	negative = ZZ.random_element(2)
	if negative == 0:
	    negOrNo = ""
	else:
	    negOrNo = "-"
	complexOrNotNumber = sample([0, 1, 2], 1)[0]
	if complexOrNotNumber==0:
	    numerator = -numeratorBefore*pi
	    denominator = denominatorBefore
	elif complexOrNotNumber==1:
	    numerator = -numeratorBefore
	    denominator = denominatorBefore*pi
	else:
	    topOrBottom = ZZ.random_element(2)
	    if topOrBottom==0:
	        numerator = -numeratorBefore
	        denominator = 0
	    else:
	        numerator = -numeratorBefore * pi
	        denominator = 0
	displayProblem = [numerator, denominator, negOrNo]
	return displayProblem
def generateDisplay(answer):
	if answer == 0:
	    numerator, denominator, negOrNo = generateNatural()
	elif answer == 1:
	    numerator, denominator, negOrNo = generateWhole()
	elif answer == 2:
	    numerator, denominator, negOrNo = generateInteger()
	elif answer == 3:
	    numerator, denominator, negOrNo = generateRational()
	elif answer == 4:
	    numerator, denominator, negOrNo = generateIrrational()
	elif answer == 5:
	    numerator, denominator, negOrNo = generateNotReal()
	else:
	    numerator, denominator, negOrNo = [0, 0, 0]
	return [numerator, denominator, negOrNo]
############ END OF DEFINITIONS ###############
listOptions=[0, 1, 2, 3, 4, 5]
###### QUESTION 1 #########
answer1 = sample(listOptions, 1)[0]
numerator1, denominator1, negOrNo1 = generateDisplay(answer1)
###### QUESTION 2 #########
answer2 = sample(listOptions, 1)[0]
numerator2, denominator2, negOrNo2 = generateDisplay(answer2)
###### QUESTION 3 #########
answer3 = sample(listOptions, 1)[0]
numerator3, denominator3, negOrNo3 = generateDisplay(answer3)
\end{sagesilent}

\textbf{Note: This part of the homework will change each time you click ``Another". You can keep clicking ``Another" to practice seeing these more difficult numbers to classify.}

\begin{exercise}
Which of the following is the \textbf{smallest} set of Real numbers that $\sage{negOrNo1} \sqrt{ \frac{\sage{numerator1}}{\sage{denominator1}}} $ belongs to?

\textit{To work around current Xronos issues, input the corresponding number for the correct set. \\
Natural - 0 \\
Whole - 1 \\
Integer - 2 \\
Rational - 3 \\
Irrational - 4 \\
Not a Real Number - 5
}

$\answer{\sage{answer1}}$

\begin{hint}
While many students have learned Irrational numbers are "Never ending, non-repeating decimals", this can be tricky with a calculator. A number like $3/17$ may look Irrational if put in a calculator, but it does end after 16 places. Your definition should include the words ``fraction" and ``integers". The best way to complete these problems are to reduce as much as possible \textbf{without making the number a decimal}.
\end{hint}
\end{exercise}

\begin{exercise}
Which of the following is the \textbf{smallest} set of Real numbers that $\sage{negOrNo2} \sqrt{\frac{\sage{numerator2}}{\sage{denominator2}}} $ belongs to?

\textit{To work around current Xronos issues, input the corresponding number for the correct set. \\
Natural - 0 \\
Whole - 1 \\
Integer - 2 \\
Rational - 3 \\
Irrational - 4 \\
Not a Real Number - 5
}

$\answer{\sage{answer2}}$
\begin{hint}
While many students have learned Irrational numbers are "Never ending, non-repeating decimals", this can be tricky with a calculator. A number like $3/17$ may look Irrational if put in a calculator, but it does end after 16 places. Your definition should include the words ``fraction" and ``integers". The best way to complete these problems are to reduce as much as possible \textbf{without making the number a decimal}.
\end{hint}
\end{exercise}

\begin{exercise}
Which of the following is the \textbf{smallest} set of Real numbers that $\sage{negOrNo3} \sqrt{\frac{\sage{numerator3}}{\sage{denominator3}}} $ belongs to?

\textit{To work around current Xronos issues, input the corresponding number for the correct set. \\
Natural - 0 \\
Whole - 1 \\
Integer - 2 \\
Rational - 3 \\
Irrational - 4 \\
Not a Real Number - 5
}

$\answer{\sage{answer3}}$
\begin{hint}
While many students have learned Irrational numbers are "Never ending, non-repeating decimals", this can be tricky with a calculator. A number like $3/17$ may look Irrational if put in a calculator, but it does end after 16 places. Your definition should include the words ``fraction" and ``integers". The best way to complete these problems are to reduce as much as possible \textbf{without making the number a decimal}.
\end{hint}
\end{exercise}

\end{document}
