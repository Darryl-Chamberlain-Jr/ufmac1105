\documentclass{ximera}
\usepackage{sagetex}
%% handout
%% space
%% newpage
%% numbers
%% nooutcomes

%% You can put user macros here
%% However, you cannot make new environments

\graphicspath{{./}{module1Activity/}{module2Activity/}{module3Activity/}}

\usepackage{sagetex}
\usepackage{tikz}
\usepackage{hyperref}
\usepackage{tkz-euclide}
\usetkzobj{all}
\pgfplotsset{compat=1.7} % prevents compile error.

\tikzstyle geometryDiagrams=[ultra thick,color=blue!50!black]
 %% we can turn off input when making a master document

\outcome{}
\author{Darryl Chamberlain Jr.}
 
\title{Objective 3 - Solving Radical Equations (Linear)}

\begin{document}
\begin{abstract}
Solve radical equations that lead to linear equations.
\end{abstract}
\maketitle

\href{https://cnx.org/contents/mwjClAV_@8.1:uI1As6DV@15/Other-Types-of-Equations}{Link to section in online textbook.}

%%%%%%%%%%%%%%%%%%%%%
%%%  Objective 3  %%%
%%%%%%%%%%%%%%%%%%%%%

You can print out \href{http://people.clas.ufl.edu/dchamberlain31/files/M5-Objective-3-Solving-Radical-Equations-that-Lead-to-Linears.pdf}{these notes} to follow along with the video below and keep notes to organize your thoughts.

\youtube{7RV_HHg_AFM}

\textbf{The major difference is in the restricted domains of radical functions!} This objective will focus on radical equations that lead to linear equations.  That means we can have 0 or 1 solution \textit{(based on whether the potential solution is in the domains of the radical functions)}. 

\begin{sagesilent}
def createSingleSolutionLinearFactors(): 
    a = (-1)**(ZZ.random_element(2))*(ZZ.random_element(2, 10))
    b = (-1)**(ZZ.random_element(2))*(ZZ.random_element(2, 10))
    c = (-1)**(ZZ.random_element(2))*(ZZ.random_element(2, 10))
    d = (-1)**(ZZ.random_element(2))*(ZZ.random_element(2, 10))
    factorA = a*x + b
    factorB = c*x + d
    answerList = solve(factorA-factorB==0, x)
    while len(answerList)==0 or b==d:
        a = (-1)**(ZZ.random_element(2))*(ZZ.random_element(2, 10))
        b = (-1)**(ZZ.random_element(2))*(ZZ.random_element(2, 10))
        c = (-1)**(ZZ.random_element(2))*(ZZ.random_element(2, 10))
        d = (-1)**(ZZ.random_element(2))*(ZZ.random_element(2, 10))
        factorA = a*x + b
        factorB = c*x + d
        answerList = solve(factorA-factorB==0, x)
    return [a, b, c, d]

a9, b9, c9, d9 = createSingleSolutionLinearFactors()
factor9A = a9*x + b9
factor9B = c9*x + d9
answer9 = round(float(solve(factor9A-factor9B==0, x)[0].rhs()), 3)
checkFactorAGreater9 = factor9A(answer9)
checkFactorBGreater9 = factor9B(answer9)
while (checkFactorAGreater9 < 0 or checkFactorBGreater9 < 0):
    a9, b9, c9, d9 = createSingleSolutionLinearFactors()
    factor9A = a9*x + b9
    factor9B = c9*x + d9
    answer9 = round(float(solve(factor9A-factor9B==0, x)[0].rhs()), 3)
    checkFactorAGreater9 = factor9A(answer9)
    checkFactorBGreater9 = factor9B(answer9)

a10, b10, c10, d10 = createSingleSolutionLinearFactors()
factor10A = a10*x + b10
factor10B = c10*x + d10
answer10 = round(float(solve(factor10A-factor10B==0, x)[0].rhs()), 3)
checkFactorAGreater10 = factor10A(answer10)
checkFactorBGreater10 = factor10B(answer10)
while (checkFactorAGreater10 < 0 or checkFactorBGreater10 < 0):
    a10, b10, c10, d10 = createSingleSolutionLinearFactors()
    factor10A = a10*x + b10
    factor10B = c10*x + d10
    answer10 = round(float(solve(factor10A-factor10B==0, x)[0].rhs()), 3)
    checkFactorAGreater10 = factor10A(answer10)
    checkFactorBGreater10 = factor10B(answer10)

a11, b11, c11, d11 = createSingleSolutionLinearFactors()
factor11A = a11*x + b11
factor11B = c11*x + d11
answer11 = round(float(solve(factor11A-factor11B==0, x)[0].rhs()), 3)
checkFactorAGreater11 = factor11A(answer11)
checkFactorBGreater11 = factor11B(answer11)
while (checkFactorAGreater11 > 0 and checkFactorBGreater11 > 0):
    a11, b11, c11, d11 = createSingleSolutionLinearFactors()
    factor11A = a11*x + b11
    factor11B = c11*x + d11
    answer11 = round(float(solve(factor11A-factor11B==0, x)[0].rhs()), 3)
    checkFactorAGreater11 = factor11A(answer11)
    checkFactorBGreater11 = factor11B(answer11)

a12, b12, c12, d12 = createSingleSolutionLinearFactors()
factor12A = a12*x + b12
factor12B = c12*x + d12
answer12 = round(float(solve(factor12A-factor12B==0, x)[0].rhs()), 3)
checkFactorAGreater12 = factor12A(answer12)
checkFactorBGreater12 = factor12B(answer12)
while (checkFactorAGreater12 > 0 and checkFactorBGreater12 > 0):
    a12, b12, c12, d12 = createSingleSolutionLinearFactors()
    factor12A = a12*x + b12
    factor12B = c12*x + d12
    answer12 = round(float(solve(factor12A-factor12B==0, x)[0].rhs()), 3)
    checkFactorAGreater12 = factor12A(answer12)
    checkFactorBGreater12 = factor12B(answer12)
\end{sagesilent}

% Q9
\begin{question}
Solve the following equation. \textit{If there is no Real solution, type "NA".}

$$ \sqrt{\sage{factor9A}} = \sqrt{\sage{factor9B}} $$

Potential Solution: $x = \answer[tolerance=0.05]{\sage{answer9}}$

Actual Solution: $x = \answer[tolerance=0.05]{\sage{answer9}}$

\begin{hint}
How can we tell if the potential solution (the value we get from squaring both sides and solving) is an actual solution? Is there anything we need to worry about when it comes to square root functions? \textit{(think domain)}
\end{hint}

\end{question}

% Q10
\begin{question}
Solve the following equation. \textit{If there is no Real solution, type "NA".}

$$ \sqrt{\sage{factor11A}} = \sqrt{\sage{factor11B}} $$

Potential Solution: $x = \answer[tolerance=0.05]{\sage{answer11}}$

Actual Solution: $x = \answer[format=string]{NA}$

\end{question}

% Q11
\begin{question}
Solve the following equation. \textit{If there is no Real solution, type "NA".}

$$ \sqrt{\sage{factor12A}} = \sqrt{\sage{factor12B}} $$

Potential Solution: $x = \answer[tolerance=0.05]{\sage{answer12}}$

Actual Solution: $x = \answer[format=string]{NA}$

\end{question}      

% Q12
\begin{question}
Solve the following equation. \textit{If there is no Real solution, type "NA".}

$$ \sqrt{\sage{factor10A}} = \sqrt{\sage{factor10B}} $$

Potential Solution: $x = \answer[tolerance=0.05]{\sage{answer10}}$

Actual Solution: $x = \answer[tolerance=0.05]{\sage{answer10}}$

\end{question}

\end{document}
