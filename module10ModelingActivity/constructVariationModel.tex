\documentclass{ximera}
\usepackage{sagetex}
%% handout
%% space
%% newpage
%% numbers
%% nooutcomes
 
%% You can put user macros here
%% However, you cannot make new environments

\graphicspath{{./}{module1Activity/}{module2Activity/}{module3Activity/}}

\usepackage{sagetex}
\usepackage{tikz}
\usepackage{hyperref}
\usepackage{tkz-euclide}
\usetkzobj{all}
\pgfplotsset{compat=1.7} % prevents compile error.

\tikzstyle geometryDiagrams=[ultra thick,color=blue!50!black]
 %% we can turn off input when making a master document
 
\outcome{}
\author{Darryl Chamberlain Jr.}
  
\title{Objective 3 - Construct Direct and Indirect Variation Models}
 
\begin{document}
\begin{abstract}

\end{abstract}

\maketitle

% Link to textbook
Link to textbook: 
\href{https://cnx.org/contents/mwjClAV_@8.12:yUH0hROr@12/Modeling-Using-Variation}{Direct and Indirect Variation.}

You can print out \href{http://people.clas.ufl.edu/dchamberlain31/files/M10M-Objective-3-Construct-Variation-Models.pdf}{these notes} to follow along with the video below and keep notes to organize your thoughts.

\youtube{dCjOPvlpIl8}

Now that we've seen how to recognize whether when we would use direct and inverse variation equations, we'll practice creating the equations. \textbf{After you complete the questions in this objective, try to describe the general process to constructing direct and inverse variation equations.}

% Question 1
\begin{sagesilent}
k = var('k')
a = var('a')
k1 = (4*pi**2) / (9.8)
a1 = ZZ.random_element(3, 9)
T1 = round(k1*sqrt(a1**3)*12, 2)
\end{sagesilent}

\begin{question}
[Astronomy] Kepler's Third Law: The square of the time, $T$, required for a planet to orbit the Sun is directly proportional to the cube of the mean distance, $a$, that the planet is from the Sun. 

\textbf{Part A.} Write the equation that describes time $T$ (years) in terms of the mean distance, $a$ (AUs). Use $k$ for your constant.

$T(a) = \answer{\sage{k}}a^{\answer{3/2}}$

\textbf{Part B.} Assume that Mars' mean distance from the Sun is $\sage{a1}$ AUs and it takes Mars about $\sage{T1}$ months to orbit the Sun. Write the equation that describes time $T$ (years) in terms of the mean distance, $a$ (AUs).

$T(a) = \answer[tolerance=0.05]{\sage{k1}}a^{\answer{3/2}}$

\begin{feedback}
\textbf{Part A.} Remember: if you initially had $c$ as a constant and took it's square root, this is \textbf{still} a constant. So we can rebrand it as $k$.

\textbf{Part B.} Check your units!
\end{feedback}
\end{question}

% Question 2
\begin{sagesilent}
P = var('P')
V = var('V')
n = var('n')
R = var('R')
T = var('T')
temp2 = ZZ.random_element(1, 6)*10
volume2 = ZZ.random_element(5, 25)
pressure2 = ZZ.random_element(1, 4)
R2 = 8.31
#
moles2B = round((pressure2 * volume2) / (temp2 * R2), 3)
coefficientB2 = R2*temp2/volume2
#
temp2C = ZZ.random_element(1, 6)*10
while temp2 == temp2C:
    temp2C = ZZ.random_element(1, 6)*10
moles2C = round((pressure2 * volume2) / (temp2C * R2), 3)
coefficientC2 = pressure2*volume2/R2
\end{sagesilent}
\begin{question}
[Chemistry] Ideal Gas Law: The product of pressure, $P$ (atmospheres), and volume, $V$ (liters), of a gas is directly proportional to the product of the amount of substance, $n$ (moles), and temperature, $T$ (Celsius). 

\textbf{Part A.} Write the equation that describes the Ideal Gas Law. Use $R$ for your constant.

$\answer{\sage{P*V}} = \answer{\sage{n*R*T}}$

\textbf{Part B.} Assume that the temperature, $\sage{temp2} C$, and volume of the gas, $\sage{volume2}$ liters, remain constant. At $\sage{pressure2}$ atmosphere of pressure, there are $\sage{moles2B}$ moles of the gas. Write an equation that describes the pressure on the gas in terms of the amount of the gas. 

$P(n) = \answer[tolerance=0.5]{\sage{coefficientB2}} n^{\answer{1}}$

\textbf{Part C.} Now assume that the pressure, $\sage{pressure2}$ atmospheres, and volume of the gas, $\sage{volume2}$ liters, remain constant. At $\sage{temp2C} C$, there are $\sage{moles2C}$ moles of the gas. Write an equation that describes the temperature of the gas in terms of the amount of the gas. 

$T(n) = \answer[tolerance=0.05]{\sage{coefficientC2}} n^{\answer{-1}}$
\end{question}

% Question 3
\begin{sagesilent}
l3 = ZZ.random_element(2, 25)
r3 = ZZ.random_element(200, 500)
k3 = l3*r3/10
\end{sagesilent}
\begin{question}
[Physics] The rate of vibration of a string under constant tension, $r$ cm/s, varies inversely with the length of the string, $l$ cm. When the string is $\sage{l3}$ mm long, the rate of vibration is $\sage{r3}$ cm/s. Write an equation that describes the rate of vibration of a string in terms of the length of the string. 

$r(l) = \answer{\sage{k3}} l^{\answer{-1}}$

\begin{hint}
Check your units!
\end{hint}
\end{question}

% Question 4
\begin{sagesilent}
tons4 = ZZ.random_element(1, 3)
mph4 = ZZ.random_element(4, 16)
energy4 = round(0.5 * (tons4 * 908) * (mph4*1609.34/3600)**2, 2)
\end{sagesilent}
\begin{question}
[Physics] The kinetic energy $K$ (J) of a moving object varies jointly with its mass $m$ (kg) and the square of its velocity $v$ (m/s). A $\sage{tons4}$ ton car traveling at $\sage{mph4}$ miles per hour has about $\sage{energy4}$ Joules of kinetic energy. Write an equation that describes the amount of kinetic energy, $K$ in terms of mass $m$ and velocity $v$.

$K = \answer[tolerance=0.05]{1/2} m^{\answer{1}} v^{\answer{2}}$

\begin{feedback}
On the exam, you would \textbf{not} be expected to know the conversions like tons to kg - these will be given on a problem-by-problem basis.
\end{feedback}

\end{question}

\end{document}