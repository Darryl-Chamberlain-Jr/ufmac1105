\documentclass{ximera}
\usepackage{sagetex}
%% handout
%% space
%% newpage
%% numbers
%% nooutcomes
 
%% You can put user macros here
%% However, you cannot make new environments

\graphicspath{{./}{module1Activity/}{module2Activity/}{module3Activity/}}

\usepackage{sagetex}
\usepackage{tikz}
\usepackage{hyperref}
\usepackage{tkz-euclide}
\usetkzobj{all}
\pgfplotsset{compat=1.7} % prevents compile error.

\tikzstyle geometryDiagrams=[ultra thick,color=blue!50!black]
 %% we can turn off input when making a master document
 
\outcome{}
\author{Darryl Chamberlain Jr.}
  
\title{Objective 1 - Limit Notation}

\begin{document}
\begin{abstract}
Interpret the notation for limits.
\end{abstract}
\maketitle
 
\href{https://cnx.org/contents/i4nRcikn@5.1:dKCfyV9u@5/The-Limit-of-a-Function}{Link to section in online textbook.}

\href{https://www.youtube.com/watch?v=tXAu5Ib-Oxk}{Intro video for limit notation}
 
%%%%%%%%%%%%%%%%%%%%%
%%%  Objective 1  %%%
%%%%%%%%%%%%%%%%%%%%%

% Limit Notation intro: Interpreting arrows

Our College Algebra textbook gives a light introduction to \href{https://cnx.org/contents/mwjClAV_@8.21:KNTP2r7D@14/Rational-Functions#Table_03_07_001}{``arrow notation"} when talking about limits. This is a great starting point to understand what exactly a limit is. 

\begin{tabular}{l | l}
\textbf{Symbol} & \textbf{Meaning} \\
\hline 
$x \rightarrow a^{-}$ & $x$ approaches $a$ from the left \\
$x \rightarrow a^{+}$ & $x$ approaches $a$ from the right \\ 
$x \rightarrow \infty$ & $x$ approaches infinity \\ 
$x \rightarrow -\infty$ & $x$ approaches negative infinity \\ 
\end{tabular}

This notation works for the output of a function as well! So if we say $f(x) \rightarrow \infty$, we mean that the output of the function approaches infinity. We've already seen this with end behavior of polynomials. For example, if we wanted to describe the end behavior of $f(x) = x^2$, we would say ``$f(x) \rightarrow \infty$ as $x \rightarrow \infty$ and as $x \rightarrow -\infty$. The limit notation condenses this phrase. 

\begin{definition}
$\lim_{x \rightarrow a} (f(x)) = L$ means ``as $x \rightarrow a$, $f(x) \rightarrow L$". 
\end{definition}

Let's practice. Use \href{https://www.desmos.com/calculator/x3mllngnj7}{this Desmos link} to answer the following questions about $f(x)=\frac{1}{x}$. 

\begin{question}
As $x$ approaches infinity, what happens to the $y$ value of $\frac{1}{x}$?

$\lim_{x \rightarrow \infty}\left(\frac{1}{x}\right) = \answer{0}$

As $x$ approaches negative infinity, what happens to the $y$ value of $\frac{1}{x}$?

$\lim_{x \rightarrow -\infty}\left(\frac{1}{x}\right) = \answer{0}$
\end{question}

Looking at the graph, you are probably wondering what we would say about the limit as $x$ approaches $0$ of $f(x) = \frac{1}{x}$. We will deal with that in the next objective. For the rest of this objective, we'll practice interpreting the limit notation. 

\begin{question}
Translate the phrase ``$\frac{x+3}{x^2-9}$ approaches $-\frac{1}{6}$ as $x$ approaches $-3$" into limit notation. 

$\lim_{\answer{x} \rightarrow \answer{-3}}\left( \answer{\sage{(x+3)/(x**2-9)}}  \right) = \answer{-\frac{1}{6}}$
\end{question}

\begin{question}
Translate the phrase ``as $x$ approaches infinity, $-(x+2)^3(x-3)^2$ approaches negative infinity" into limit notation. 
	
$\lim_{\answer{x} \rightarrow \answer{\sage{Infinity} } }\left( \answer{\sage{-(x+2)**3 * (x-3)**2}}  \right) = \answer{\sage{-Infinity}}$
\end{question}

\begin{question}
Translate the phrase ``as $x$ approaches $3$, $\frac{1}{(x-3)^2}$ approaches infinity" into limit notation. 
	
$\lim_{\answer{x} \rightarrow \answer{3}}\left( \answer{\sage{1/(x-3)**2}}  \right) = \answer{\sage{Infinity}}$
\end{question}

\end{document}