\documentclass{ximera}
\usepackage{sagetex}
%% handout
%% space
%% newpage
%% numbers
%% nooutcomes

%% You can put user macros here
%% However, you cannot make new environments

\graphicspath{{./}{module1Activity/}{module2Activity/}{module3Activity/}}

\usepackage{sagetex}
\usepackage{tikz}
\usepackage{hyperref}
\usepackage{tkz-euclide}
\usetkzobj{all}
\pgfplotsset{compat=1.7} % prevents compile error.

\tikzstyle geometryDiagrams=[ultra thick,color=blue!50!black]
 %% we can turn off input when making a master document

\outcome{Recognize and construct linear functions as well as solve linear equations.}
\author{Darryl Chamberlain Jr.}
 
\title{Objective 4 - Solve Linear Equations}

\begin{document}
\begin{abstract}
Solve linear equations. 
\end{abstract}
\maketitle

\href{https://cnx.org/contents/mwjClAV_@8.1:62_eXnY6@14/Linear-Equations-in-One-Variable}{Link to section in online textbook.}

%%%%%%%%%%%%%%%%%%%%%
%%%  Objective 2  %%%
%%%%%%%%%%%%%%%%%%%%%

Now, watch \underline{\href{https://mediasite.video.ufl.edu/Mediasite/Play/486c7ecd0ea14369bfc405492ae942f51d}{this video}} to review how to solve linear equations. These techniques will be used throughout most of the semester. Be sure to write notes to yourself that you can review later!

Now try to solve the following linear equations. 

\begin{sagesilent}
# SOLVES BASIC LINEAR EQUATIONS OF THE FORM
    # b0 * (b1 + b2 * x) = b3 ( x * b4 - b5)

x = var('x')

###################
def maybeMakeNegative(rational):
    maybeNegative = (-1)**ZZ.random_element(2)
    rational = maybeNegative * rational
    return rational

def generateBlocks():
    listNaturals = range(2, 16)
    n0, n1, n2, n3, n4, n5 = sample(listNaturals, 6)
    b0 = Integer(-n0)
    b1 = Integer(maybeMakeNegative(n1))
    b2 = Integer(maybeMakeNegative(n2))
    b3 = Integer(-n3)
    b4 = Integer(maybeMakeNegative(n4))
    b5 = Integer(maybeMakeNegative(n5))
    # Begin checking for one solution
    OneSolutionCheck = b0*b2 - b3*b4
    # Makes sure there is exactly one solution
    while (OneSolutionCheck == 0):
        listNaturals = range(2, 16)
        n0, n1, n2, n3, n4, n5 = sample(listNaturals, 6)
        b0 = Integer(-n0)
        b1 = Integer(maybeMakeNegative(n1))
        b2 = Integer(maybeMakeNegative(n2))
        b3 = Integer(-n3)
        b4 = Integer(maybeMakeNegative(n4))
        b5 = Integer(maybeMakeNegative(n5))
        # Begin checking for one solution
        OneSolutionCheck = b0*b2 - b3*b4
    blocks = [b0, b1, b2, b3, b4, b5]
    return blocks

#b0 * (b1 + b2 * x) =  b3 ( x * b4 - b5)
def generateSolution(blocks):
    a, b, c, d, e, f = blocks
    basicLinearEquation = a * (b + c * x) - d * ( x * e - f)
    solution = float(solve(basicLinearEquation, x)[0].rhs())
    return solution
    
def generateFeedback(blocks):
    a, b, c, d, e, f = blocks
    basicLinearEquation = a * (b - c * x) - d * ( x * e + f)
    feedback = float(solve(basicLinearEquation, x)[0].rhs())
    return feedback

######### END OF DEFINITIONS ##########

##### QUESTION 1 #####
blocks1 = generateBlocks()
answer1 = generateSolution(blocks1)
feedback1 = generateFeedback(blocks1)
displayEquationLeft1 = blocks1[1]+blocks1[2]*x
displayEquationRight1 = blocks1[4]*x-blocks1[5]

##### QUESTION 2 #####
blocks2 = generateBlocks()
answer2 = generateSolution(blocks2)
feedback2 = generateFeedback(blocks2)
displayEquationLeft2 = blocks2[1]+blocks2[2]*x
displayEquationRight2 = blocks2[4]*x-blocks2[5]

##### QUESTION 3 #####
blocks3 = generateBlocks()
answer3 = generateSolution(blocks3)
feedback3 = generateFeedback(blocks3)
displayEquationLeft3 = blocks3[1]+blocks3[2]*x
displayEquationRight3 = blocks3[4]*x-blocks3[5]

\end{sagesilent}

\begin{exercise}
Solve the equation below. 

$\sage{blocks1[0]} (\sage{displayEquationLeft1}) = \sage{blocks1[3]} (\sage{displayEquationRight1})$

$x = \answer[tolerance=0.01]{\sage{answer1}}$

\begin{feedback}[incorrect]
	Did you get $\sage{feedback1}$ as your answer? If so, you are not distributing correctly. Remember that if you have something like $-3(2x-1)$, you need to distribute the $-3$ to BOTH $2x$ and $-1$ to get $-6x+3$. 
\end{feedback}
\end{exercise}

\begin{exercise}
Solve the equation below. 

$\sage{blocks2[0]} (\sage{displayEquationLeft2}) = \sage{blocks2[3]} (\sage{displayEquationRight2})$

$x = \answer[tolerance=0.01]{\sage{answer2}}$

\begin{feedback}[incorrect]
	Did you get $\sage{feedback2}$ as your answer? If so, you are not distributing correctly. Remember that if you have something like $-3(2x-1)$, you need to distribute the $-3$ to BOTH $2x$ and $-1$ to get $-6x+3$. 
\end{feedback}
\end{exercise}

\begin{exercise}
Solve the equation below. 

$\sage{blocks3[0]} (\sage{displayEquationLeft3}) = \sage{blocks3[3]} (\sage{displayEquationRight3})$

$x = \answer[tolerance=0.01]{\sage{answer3}}$

\begin{feedback}[incorrect]
	Did you get $\sage{feedback3}$ as your answer? If so, you are not distributing correctly. Remember that if you have something like $-3(2x-1)$, you need to distribute the $-3$ to BOTH $2x$ and $-1$ to get $-6x+3$. 
\end{feedback}
\end{exercise}

%%%%%%%%%% TYPE 2 - ADVANCED LINEAR EQUATIONS

\begin{sagesilent}
# Type 2 - Solve Advanced linear equations (fractions)
#(coefficients[0]*x + numerators[0])/denominators[0]
    # - (coefficients[1]*x+numerators[1])/denominators[1]
    # = (coefficients[2]*x+numerators[2])/denominators[2]

x = var('x')
###################
def maybeMakeNegative(rational):
    maybeNegative = (-1)**ZZ.random_element(2)
    rational = maybeNegative * rational
    return rational

#No restrictions on coefficients or numerators
def createThreeRandomIntegers():
    a = maybeMakeNegative(ZZ.random_element(3, 9))
    b = maybeMakeNegative(ZZ.random_element(3, 9))
    c = maybeMakeNegative(ZZ.random_element(3, 9))
    return [a, b, c]

def createThreeDistinctRandomNaturals():
    possibleNaturals= range(2,7)
    n1, n2, n3 = sample(possibleNaturals, 3)
    naturals = [Integer(n1), Integer(n2), Integer(n3)]
    return naturals

def createThreeDistinctRandomIntegers():
    a, b, c = sample(range(3, 8), 3)
    return [Integer(maybeMakeNegative(a)), Integer(maybeMakeNegative(b)), Integer(maybeMakeNegative(c))]

def createViableConstants():
    a, b, c = createThreeRandomIntegers()
    d, e, f = createThreeRandomIntegers()
    g, h, i = createThreeDistinctRandomNaturals()
    # Check that there is exactly one solution to the linear equation
    OneSolutionCheck = (a/g) - (b/h) - (c/i)
    while (OneSolutionCheck == 0):
        a, b, c = createThreeRandomIntegers()
        d, e, f = createThreeRandomIntegers()
        g, h, i = createThreeDistinctRandomNaturals()
        OneSolutionCheck = (a/g) - (b/h) - (c/i)
    return [a, b, c, d, e, f, g, h, i]

def createSolution(constants):
    a, b, c, d, e, f, g, h, i = constants
    equationBlockOne = (a*x+d)/g
    equationBlockTwo = (b*x+e)/h
    equationBlockThree = (c*x+f)/i
    toSolve = equationBlockOne - equationBlockTwo - equationBlockThree
    solution = round(float(solve(toSolve, x)[0].rhs()), 3)
    return solution
    
def generateFeedback1(constants):
    a, b, c, d, e, f, g, h, i = constants
    equationBlockOne = (a*x+d)/g
    equationBlockTwo = (b*x-e)/h
    equationBlockThree = (c*x+f)/i
    toSolve = equationBlockOne - equationBlockTwo - equationBlockThree
    feedback = round(float(solve(toSolve, x)[0].rhs()), 3)
    return feedback
    
def generateFeedback2(constants):
    a, b, c, d, e, f, g, h, i = constants
    equationBlockOne = (a*x)/g + d
    equationBlockTwo = (b*x)/h + e
    equationBlockThree = (c*x)/i + f
    toSolve = equationBlockOne - equationBlockTwo - equationBlockThree
    feedback = round(float(solve(toSolve, x)[0].rhs()), 3)
    return feedback

def generateFeedback3(constants): 
    a, b, c, d, e, f, g, h, i = constants
    equationBlockOne = (a*x)/g + d
    equationBlockTwo = (b*x)/h - e
    equationBlockThree = (c*x)/i + f
    toSolve = equationBlockOne - equationBlockTwo - equationBlockThree
    feedback = round(float(solve(toSolve, x)[0].rhs()), 3)
    return feedback


######## END OF DEFINITIONS ###########

##### QUESTION 4 #####
constants4 = createViableConstants()
displayNumeratorA4 = constants4[0] * x + constants4[3]
displayNumeratorB4 = constants4[1] * x + constants4[4]
displayNumeratorC4 = constants4[2] * x + constants4[5]
answer4 = createSolution(constants4)
feedback41 = generateFeedback1(constants4)
feedback42 = generateFeedback2(constants4)
feedback43 = generateFeedback3(constants4)

##### QUESTION 5 #####
constants5 = createViableConstants()
displayNumeratorA5 = constants5[0] * x + constants5[3]
displayNumeratorB5 = constants5[1] * x + constants5[4]
displayNumeratorC5 = constants5[2] * x + constants5[5]
answer5 = createSolution(constants5)
feedback51 = generateFeedback1(constants5)
feedback52 = generateFeedback2(constants5)
feedback53 = generateFeedback3(constants5)

##### QUESTION 6 #####
constants6 = createViableConstants()
displayNumeratorA6 = constants6[0] * x + constants6[3]
displayNumeratorB6 = constants6[1] * x + constants6[4]
displayNumeratorC6 = constants6[2] * x + constants6[5]
answer6 = createSolution(constants6)
feedback61 = generateFeedback1(constants6)
feedback62 = generateFeedback2(constants6)
feedback63 = generateFeedback3(constants6)
\end{sagesilent}

\begin{exercise}
Solve the equation below. 

$\frac{\sage{displayNumeratorA4}}{\sage{constants4[6]}} - \frac{\sage{displayNumeratorB4}}{\sage{constants4[7]}} = \frac{\sage{displayNumeratorC4}}{\sage{constants4[8]}}$

\begin{hint}
Adding/Multiplying fractions can be difficult and tedious. Is there something we can multiply both sides of the equation by to remove the fractions from the equation?
\end{hint}

$x = \answer[tolerance=0.01]{\sage{answer4}}$

\begin{feedback}
	There are two common issues when solving linear equations:
	
	The first we saw in the previous set of problems: not distributing correctly. If you got $\sage{feedback41}$, check that you distributed any negatives correctly. 
	
	The second common issue is not dividing correctly. If we have a fraction like $\frac{6x-4}{2}$, the 2 is dividing \textbf{both} parts. So, this would become $3x-2$ and not $3x-4$. If you got $\sage{feedback42}$, you made this type of mistake. 
	
	If you made both of these mistakes, you got $\sage{feedback43}$. Restart the problem and correct both issues. 
\end{feedback}
\end{exercise}

\begin{exercise}
Solve the equation below. 

$\frac{\sage{displayNumeratorA5}}{\sage{constants5[6]}} - \frac{\sage{displayNumeratorB5}}{\sage{constants5[7]}} = \frac{\sage{displayNumeratorC5}}{\sage{constants5[8]}}$

$x = \answer[tolerance=0.01]{\sage{answer5}}$

\begin{feedback}
	There are two common issues when solving linear equations:
	
	The first we saw in the previous set of problems: not distributing correctly. If you got $\sage{feedback51}$, check that you distributed any negatives correctly. 
	
	The second common issue is not dividing correctly. If we have a fraction like $\frac{6x-4}{2}$, the 2 is dividing \textbf{both} parts. So, this would become $3x-2$ and not $3x-4$. If you got $\sage{feedback52}$, you made this type of mistake. 
	
	If you made both of these mistakes, you got $\sage{feedback53}$. Restart the problem and correct both issues. 
\end{feedback}
\end{exercise}

\begin{exercise}
Solve the equation below. 

$\frac{\sage{displayNumeratorA6}}{\sage{constants6[6]}} - \frac{\sage{displayNumeratorB6}}{\sage{constants6[7]}} = \frac{\sage{displayNumeratorC6}}{\sage{constants6[8]}}$

$x= \answer[tolerance=0.01]{\sage{answer6}}$

\begin{feedback}
	There are two common issues when solving linear equations:
	
	The first we saw in the previous set of problems: not distributing correctly. If you got $\sage{feedback61}$, check that you distributed any negatives correctly. 
	
	The second common issue is not dividing correctly. If we have a fraction like $\frac{6x-4}{2}$, the 2 is dividing \textbf{both} parts. So, this would become $3x-2$ and not $3x-4$. If you got $\sage{feedback62}$, you made this type of mistake. 
	
	If you made both of these mistakes, you got $\sage{feedback63}$. Restart the problem and correct both issues. 
\end{feedback}
\end{exercise}

\end{document}