\documentclass{ximera}
\usepackage{sagetex}
%% handout
%% space
%% newpage
%% numbers
%% nooutcomes

%% You can put user macros here
%% However, you cannot make new environments

\graphicspath{{./}{module1Activity/}{module2Activity/}{module3Activity/}}

\usepackage{sagetex}
\usepackage{tikz}
\usepackage{hyperref}
\usepackage{tkz-euclide}
\usetkzobj{all}
\pgfplotsset{compat=1.7} % prevents compile error.

\tikzstyle geometryDiagrams=[ultra thick,color=blue!50!black]
 %% we can turn off input when making a master document

\outcome{Recognize and construct linear functions as well as solve linear equations.}
\author{Darryl Chamberlain Jr.}
\title{Objective 1 - Construct a linear function from points}

\begin{document}
\begin{abstract}
Use points to construct a linear function. 
\end{abstract}
\maketitle

\href{https://cnx.org/contents/mwjClAV_@8.1:62_eXnY6@14/Linear-Equations-in-One-Variable}{Link to section in online textbook.}

%%%%%%%%%%%%%%%%%%%%%
%%%  Objective 1  %%%
%%%%%%%%%%%%%%%%%%%%%

First, watch \underline{\href{https://mediasite.video.ufl.edu/Mediasite/Play/ea9a27e045aa4cbeb9211f7d64e96cb31d}{this video}} to learn about what is necessary to construct a linear function.

This objective will focus on constructing linear functions from a point and slope or from two points.

\begin{sagesilent}
x = var('x')

#################
def maybeMakeNegative(rational):
    maybeNegative = (-1)**ZZ.random_element(2)
    rational = maybeNegative * rational
    return rational

def generatePoints():
    pointOne = [maybeMakeNegative(ZZ.random_element(2,9)), maybeMakeNegative(ZZ.random_element(2, 9))]
    pointTwo = [maybeMakeNegative(ZZ.random_element(2, 9)), maybeMakeNegative(ZZ.random_element(2, 9))]
    # Makes sure coordinates are distinct and not a vertical/horizontal line
    while (pointOne[0]==pointTwo[0] or pointOne[1]==pointTwo[1]):
        pointOne = [maybeMakeNegative(ZZ.random_element(2, 9)), maybeMakeNegative(ZZ.random_element(2, 9))]
        pointTwo = [maybeMakeNegative(ZZ.random_element(2, 9)), maybeMakeNegative(ZZ.random_element(2, 9))]
    return [pointOne, pointTwo]

def generateSolution(problem):
    pointOne, pointTwo = problem
    numerator = pointTwo[1]-pointOne[1]
    denominator = pointTwo[0]-pointOne[0]
    slope = float(numerator/denominator)
    yInt = float(pointTwo[1]-slope*pointTwo[0])
    slope = slope
    #print "Slope %s and y-Intercept %s" %(slope, yInt)
    return [slope, yInt]

######## END OF DEFINITIONS #########

##### QUESTION 1 #####
points1 = generatePoints()
pointOne1, pointTwo1 = points1
solution1 = generateSolution(points1)
equation1 = solution1[0]*x + solution1[1]

##### QUESTION 2 #####
points2 = generatePoints()
pointOne2, pointTwo2 = points2
solution2 = generateSolution(points2)
equation2 = solution2[0]*x + solution2[1]

##### QUESTION 3 #####
points3 = generatePoints()
pointOne3, pointTwo3 = points3
solution3 = generateSolution(points3)
equation3 = solution3[0]*x + solution3[1]
\end{sagesilent}

\begin{question}
Find the equation of the line containing the two points below. Write the equation in slope-intercept form. \\
$(\sage{pointOne1[0]}, \sage{pointOne1[1]}) \text{ and } (\sage{pointTwo1[0]}, \sage{pointTwo1[1]})$

$y = \answer[tolerance=0.05]{\sage{solution1[0]}}x$ $+ \answer[tolerance=0.05]{\sage{solution1[1]}}$ 

\begin{hint}
To construct a linear function, we need its slope and a single point on the line. Can we figure out the slope from two points?
\end{hint}
\end{question}

\begin{question}
Find the equation of the line containing the two points below. Write the equation in slope-intercept form. \\
$(\sage{pointOne2[0]}, \sage{pointOne2[1]}) \text{ and } (\sage{pointTwo2[0]}, \sage{pointTwo2[1]})$

$y = \answer[tolerance=0.05]{\sage{solution2[0]}}x$ $+ \answer[tolerance=0.05]{\sage{solution2[1]}}$ 
\end{question}

\begin{question}
Find the equation of the line containing the two points below. Write the equation in slope-intercept form. \\
$(\sage{pointOne3[0]}, \sage{pointOne3[1]}) \text{ and } (\sage{pointTwo3[0]}, \sage{pointTwo3[1]})$

$y = \answer[tolerance=0.05]{\sage{solution3[0]}}x$ $+ \answer[tolerance=0.05]{\sage{solution3[1]}}$ 
\end{question}

%%%%%%%%% TYPE 2 - PARALLEL OR PERPENDICULAR

\begin{sagesilent}
x, y = var('x, y')

#################
def maybeMakeNegative(rational):
    maybeNegative = (-1)**ZZ.random_element(2)
    rational = maybeNegative * rational
    return rational

def generateProblemAndSolution(lineType):
    if lineType == 0:
        point = [maybeMakeNegative(ZZ.random_element(2, 11)), maybeMakeNegative(ZZ.random_element(2, 11))]
        aCoeff = ZZ.random_element(3, 10)
        bCoeff = maybeMakeNegative(ZZ.random_element(3, 10))
        cCoeff = ZZ.random_element(3, 16)
        # Slope and y-intercept
        slope = float(-aCoeff/bCoeff)
        yInt = float(point[1]-slope*point[0])
        return [[slope, yInt], point, [aCoeff, bCoeff, cCoeff]]
    else:
        point = [maybeMakeNegative(ZZ.random_element(2, 11)), maybeMakeNegative(ZZ.random_element(2, 11))]
        aCoeff = ZZ.random_element(3, 10)
        bCoeff = maybeMakeNegative(ZZ.random_element(3, 10))
        cCoeff = ZZ.random_element(3, 16)
        # Slope and y-intercept
        slope = float(bCoeff/aCoeff)
        yInt = float(point[1]-slope*point[0])
        return [[slope, yInt], point, [aCoeff, bCoeff, cCoeff]]

################# QUESTION 4 - Parallel #####################
solution4, point4, coefficients4 = generateProblemAndSolution(0)
aCoeff4, bCoeff4, cCoeff4 = coefficients4
standardForm4 = aCoeff4 * x + bCoeff4 * y
pointSlopeForm4 = solution4[0]*x + solution4[1]

################# QUESTION 5 - Parallel #####################
solution5, point5, coefficients5 = generateProblemAndSolution(0)
aCoeff5, bCoeff5, cCoeff5 = coefficients5
standardForm5 = aCoeff5 * x + bCoeff5 * y
pointSlopeForm5 = solution5[0]*x + solution5[1]

################# QUESTION 6 - Perpendicular #####################
solution6, point6, coefficients6 = generateProblemAndSolution(1)
aCoeff6, bCoeff6, cCoeff6 = coefficients6
standardForm6 = aCoeff6 * x + bCoeff6 * y
pointSlopeForm6 = solution6[0]*x + solution6[1]

################# QUESTION 7 - Perpendicular #####################
solution7, point7, coefficients7 = generateProblemAndSolution(1)
aCoeff7, bCoeff7, cCoeff7 = coefficients7
standardForm7 = aCoeff7 * x + bCoeff7 * y
pointSlopeForm7 = solution7[0]*x + solution7[1]
\end{sagesilent}

\textbf{For these problems, you'll be given a description of the line and a point. Think about what information you should get from the line, then use the point to construct a new linear function.}

\begin{question}
Find the equation of the line described below. Write the equation of the line in slope-intercept form. \\

Parallel to $\sage{standardForm4}=\sage{cCoeff4}$ and passing through the point $(\sage{point4[0]}, \sage{point4[1]})$.

$y = \answer[tolerance=0.05]{\sage{solution4[0]}}x$ $+ \answer[tolerance=0.05]{\sage{solution4[1]}}$ 

\begin{hint}
If a line is parallel to another, what does that mean about its slope?
\end{hint}
\end{question}

\begin{question}
Find the equation of the line described below. Write the equation of the line in slope-intercept form. \\

Parallel to $\sage{standardForm5}=\sage{cCoeff5}$ and passing through the point $(\sage{point5[0]}, \sage{point5[1]})$.

$y = \answer[tolerance=0.05]{\sage{solution5[0]}}x$ $+ \answer[tolerance=0.05]{\sage{solution5[1]}}$ 
\end{question}

\begin{question}
Find the equation of the line described below. Write the equation of the line in slope-intercept form. \\

Perpendicular to $\sage{standardForm6}=\sage{cCoeff6}$ and passing through the point $(\sage{point6[0]}, \sage{point6[1]})$.

$y = \answer[tolerance=0.05]{\sage{solution6[0]}}x$ $+ \answer[tolerance=0.05]{\sage{solution6[1]}}$ 
\begin{hint}
If a line is perpendicular to another, what does that mean about its slope?
\end{hint}
\end{question}

\begin{question}
Find the equation of the line described below. Write the equation of the line in slope-intercept form. \\

Perpendicular to $\sage{standardForm7}=\sage{cCoeff7}$ and passing through the point $(\sage{point7[0]}, \sage{point7[1]})$.

$y = \answer[tolerance=0.05]{\sage{solution7[0]}}x$ $+ \answer[tolerance=0.05]{\sage{solution7[1]}}$ 
\end{question}

\end{document}