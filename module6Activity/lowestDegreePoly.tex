\documentclass{ximera}
\usepackage{sagetex}
%% handout
%% space
%% newpage
%% numbers
%% nooutcomes

%% You can put user macros here
%% However, you cannot make new environments

\graphicspath{{./}{module1Activity/}{module2Activity/}{module3Activity/}}

\usepackage{sagetex}
\usepackage{tikz}
\usepackage{hyperref}
\usepackage{tkz-euclide}
\usetkzobj{all}
\pgfplotsset{compat=1.7} % prevents compile error.

\tikzstyle geometryDiagrams=[ultra thick,color=blue!50!black]
 %% we can turn off input when making a master document

\outcome{}
\author{Darryl Chamberlain Jr.}

\title{Objective 3 - Lowest-Degree Polynomial}

\begin{document}
\begin{abstract}
Construct a lowest-degree polynomial given its zeros.
\end{abstract}
\maketitle

\href{https://cnx.org/contents/mwjClAV_@8.1:G7V9LHif@17/Zeros-of-Polynomial-Functions}{Link to section in online textbook}

%%%%%%%%%%%%%%%%%%%%%
%%%  Objective 3  %%%
%%%%%%%%%%%%%%%%%%%%%

You can print out \href{http://people.clas.ufl.edu/dchamberlain31/files/M6-Objective-3-Construct-Polynomials.pdf}{these notes} to follow along with the video below and keep notes to organize your thoughts.

\youtube{g3cpbm3fQjw}

Now practice constructing polynomials from zeros with the questions below.

\begin{sagesilent}
def maybeMakeNegative(natural):
    integer = (-1)**ZZ.random_element(2) * natural
    return integer

def generateFactorEasy():
    a = 1
    b = maybeMakeNegative(ZZ.random_element(3, 6))
    return [a, b]

def generateFactorHard():
    a = ZZ.random_element(1, 4)
    b = maybeMakeNegative(ZZ.random_element(3, 6))
    while gcd(a, b) > 1:
        a = ZZ.random_element(1, 4)
        b = maybeMakeNegative(ZZ.random_element(3, 6))
    return [a, b]

def generateAllFactorsEasy():
    factor1 = generateFactorEasy()
    factor2 = generateFactorEasy()
    factor3 = generateFactorEasy()
    return [factor1, factor2, factor3]

def generateAllFactorsHard():
    factor1 = generateFactorHard()
    factor2 = generateFactorHard()
    factor3 = generateFactorHard()
    return [factor1, factor2, factor3]

def generateZeros(factor1, factor2, factor3):
    zero1 = factor1[1]/factor1[0]
    zero2 = factor2[1]/factor2[0]
    zero3 = factor3[1]/factor3[0]
    return [zero1, zero2, zero3]

def generatePolynomial(factor1, factor2, factor3):
    f10, f11 = factor1
    f20, f21 = factor2
    f30, f31 = factor3
    #
    a = f10*f20*f30
    b = -f11*f20*f30 - f10*f21*f30 - f10*f20*f31
    c = f11*f20*f31 + f10*f21*f31 + f11*f21*f30
    d = -f11*f21*f31
    #
    coefficients = [a, b, c, d]
    return coefficients

##### QUESTION 13 #####
factor131, factor132, factor133 = generateAllFactorsEasy()
zero131, zero132, zero133 = generateZeros(factor131, factor132, factor133)
coefficients13 = generatePolynomial(factor131, factor132, factor133)

##### QUESTION 14 #####
factor141, factor142, factor143 = generateAllFactorsEasy()
zero141, zero142, zero143 = generateZeros(factor141, factor142, factor143)
coefficients14 = generatePolynomial(factor141, factor142, factor143)

##### QUESTION 15 #####
factor151, factor152, factor153 = generateAllFactorsHard()
zero151, zero152, zero153 = generateZeros(factor151, factor152, factor153)
coefficients15 = generatePolynomial(factor151, factor152, factor153)

##### QUESTION 16 #####
factor161, factor162, factor163 = generateAllFactorsHard()
zero161, zero162, zero163 = generateZeros(factor161, factor162, factor163)
coefficients16 = generatePolynomial(factor161, factor162, factor163)
\end{sagesilent}

%Q 13
\begin{question}
Construct the lowest-degree polynomial given the zeros below.

$$ \sage{zero131}, \sage{zero132}, \sage{zero133} $$

$f(x) = \answer{\sage{coefficients13[0]}} x^3 + \answer{\sage{coefficients13[1]}} x^2 + \answer{\sage{coefficients13[2]}} x + \answer{\sage{coefficients13[3]}}$

\end{question}

%Q 14
\begin{question}
Construct the lowest-degree polynomial given the zeros below.

$$ \sage{zero141}, \sage{zero142}, \sage{zero143} $$

$f(x) = \answer{\sage{coefficients14[0]}} x^3 + \answer{\sage{coefficients14[1]}} x^2 + \answer{\sage{coefficients14[2]}} x + \answer{\sage{coefficients14[3]}}$

\end{question}

%Q 15
\begin{question}
Construct the lowest-degree polynomial given the zeros below.

$$ \sage{zero151}, \sage{zero152}, \sage{zero153} $$

$f(x) = \answer{\sage{coefficients15[0]}} x^3 + \answer{\sage{coefficients15[1]}} x^2 + \answer{\sage{coefficients15[2]}} x + \answer{\sage{coefficients15[3]}}$

\begin{hint}
Remember back to what it meant to be in Standard Form for linear functions: we did not have any fractions as coefficients. How would we rewrite a factor that has a fraction in it, like $\left(x-\frac{3}{4}\right)$?
\end{hint}

\end{question}

%Q 16
\begin{question}
Construct the lowest-degree polynomial given the zeros below.

$$ \sage{zero161}, \sage{zero162}, \sage{zero163} $$

$f(x) = \answer{\sage{coefficients16[0]}} x^3 + \answer{\sage{coefficients16[1]}} x^2 + \answer{\sage{coefficients16[2]}} x + \answer{\sage{coefficients16[3]}}$

\begin{hint}
Remember back to what it meant to be in Standard Form for linear functions: we did not have any fractions as coefficients. How would we rewrite a factor that has a fraction in it, like $\left(x-\frac{3}{4}\right)$?
\end{hint}

\end{question}

We focused on building polynomials with integer and rational zeros. What would we do if we had other types of zeros, like irrational or complex?

\begin{theorem}
Complex and Irrational roots for polynomials come in ``\_\_\_\_\_\_\_\_\_" pairs. \\

The Quadratic Formula

$$x = \frac{-b \pm \sqrt{b^2 - 4ac}}{2a}$$

tells us something about the types of zeros a quadratic function may have:
	\begin{itemize}
		\item 2 different, rational zeros
			\begin{itemize}
				\item $\text{e.g., } \frac{1}{4} \text{ and } -3 \text{ for the polynomial } 4x^2+11x-3$
			\end{itemize}
		\item 2 copies of a rational zero
			\begin{itemize}
				\item $\text{e.g., } \frac{1}{3} \text{ and } \frac{1}{3} \text{ for the polynomial } 9x^2-2x+1$
			\end{itemize}
		\item 2 different, irrational zeros
			\begin{itemize}
				\item $\text{e.g., } \frac{1}{2} - \sqrt{2} \text{ and } \frac{1}{2} + \sqrt{2} \text{ for the polynomial } 4x^2-4x-7$
			\end{itemize}
		\item 2 different, complex zeros
			\begin{itemize}
				\item $\text{e.g., } \frac{1}{4}-3i \text{ and } \frac{1}{4}+3i \text{ for the polynomial } 16x^2-8x+145$
			\end{itemize}
	\end{itemize}

Let's focus on the irrational and complex zeros. These occur when the number under the square root is either (1) not a perfect square or (2) negative. Let's look closer at the form these zeros take by looking at the subgroups the numbers belong to.

Case 1: $b^2 - 4ac$ is positive and is \textbf{not} a perfect square.

$$x = \frac{\text{integer}}{\text{integer}} \pm \frac{\text{irrational}}{\text{integer}}$$

$$ x = \text{rational} \pm \text{irrational} $$

Case 2: $b^2 - 4ac$ is negative.

$$x = \frac{\text{integer}}{\text{integer}} \pm \frac{\text{complex}}{\text{integer}}$$

$$ x = \text{rational} \pm \text{complex} $$
\end{theorem}

\begin{question}
What word describes the relationship between the zeros $ x = \text{rational} - \text{complex} $ and $ x = \text{rational} + \text{complex}$?

They are $\answer[format=string]{conjugate}$ pairs!

\begin{hint}
What are $3+4i$ and $3-4i$ to each other?
\end{hint}

\end{question}

We use this theorem to construct polynomials with irrational and/or complex roots.

\begin{sagesilent}
def constructLowestDegreePolyIrrational(b):
    a = ZZ.random_element(2, 8)
    n = ZZ.random_element(-5, 5)
    d = ZZ.random_element(2, 8)
    while gcd(abs(n), d) > 1 or n == 0:
        n = ZZ.random_element(-5, 5)
        d = ZZ.random_element(2, 8)
    coefficients = [d, -2*b*d-n, 2*b*n + b**2*d - a*d, -b**2*n+a*n]
    zeros = [b, sqrt(a), n/d]
    return [coefficients, zeros]

def constructLowestDegreePolyComplex(a):
    b = ZZ.random_element(2, 8)
    n = ZZ.random_element(-5, 5)
    d = ZZ.random_element(2, 8)
    while gcd(abs(n), d) > 1 or n == 0:
        n = ZZ.random_element(-5, 5)
        d = ZZ.random_element(2, 8)
    coefficients = [d, -n-2*a*d, 2*a*n+a**2*d+b**2*d, -a**2*n-b**2*n]
    zeros = [a, b, n/d]
    hintValues = [n, d]
    return [coefficients, zeros, hintValues]
#
irrationalB = ZZ.random_element(2, 6)
complexA = ZZ.random_element(2, 6)
#
# Q6 - Irrational Roots with b=0
coefficientsQ6, zerosQ6 = constructLowestDegreePolyIrrational(0)
# Q7 - Irrational Roots with b neq 0
coefficientsQ7, zerosQ7 = constructLowestDegreePolyIrrational(irrationalB)
# Q8 - Complex Roots with a=0
coefficientsQ8, zerosQ8, hintValuesQ8 = constructLowestDegreePolyComplex(0)
# Q9 - Complex Roots with a neq 0
coefficientsQ9, zerosQ9, hintValuesQ9 = constructLowestDegreePolyComplex(complexA)
\end{sagesilent}

% Irrational roots - easy
\begin{question}
Construct the lowest-degree polynomial given the zeros below.

$$\sage{zerosQ6[1]}, \sage{zerosQ6[2]}$$

$f(x) = \answer{\sage{coefficientsQ6[0]}}x^3 + \answer{\sage{coefficientsQ6[1]}}x^2 + \answer{\sage{coefficientsQ6[2]}}x + \answer{\sage{coefficientsQ6[3]}}$

\begin{hint}
If $\sage{zerosQ6[1]}$ is a zero to the polynomial, then $-\sage{zerosQ6[1]}$ is also! Multiply $(x-\sage{zerosQ6[1]})(x+\sage{zerosQ6[1]})$ first, then use the third zero to finish building the polynomial.
\end{hint}

\end{question}

% Irrational roots - hard
\begin{question}
Construct the lowest-degree polynomial given the zeros below.

$$\sage{zerosQ7[0]}+\sage{zerosQ7[1]}, \sage{zerosQ7[2]}$$

$f(x) = \answer{\sage{coefficientsQ7[0]}}x^3 + \answer{\sage{coefficientsQ7[1]}}x^2 + \answer{\sage{coefficientsQ7[2]}}x + \answer{\sage{coefficientsQ7[3]}}$

\begin{hint}
Be careful with how you set up this problem. Again, multiply the conjugate factors together first. If you did this right, there should be no radicals left!
\end{hint}

\end{question}

% Complex roots - easy
\begin{question}
Construct the lowest-degree polynomial given the zeros below.

$$\sage{zerosQ8[1]}i, \sage{zerosQ8[2]}$$

$f(x) = \answer{\sage{coefficientsQ8[0]}}x^3 + \answer{\sage{coefficientsQ8[1]}}x^2 + \answer{\sage{coefficientsQ8[2]}}x + \answer{\sage{coefficientsQ8[3]}}$
\end{question}

% Complex roots - hard
\begin{question}
Construct the lowest-degree polynomial given the zeros below.

$$\sage{zerosQ9[0]}+\sage{zerosQ9[1]}i, \sage{zerosQ9[2]}$$

$f(x) = \answer{\sage{coefficientsQ9[0]}}x^3 + \answer{\sage{coefficientsQ9[1]}}x^2 + \answer{\sage{coefficientsQ9[2]}}x + \answer{\sage{coefficientsQ9[3]}}$

\begin{hint}
First, you want to set up the factors:

$$(x - \, (\sage{zerosQ9[0]}+\sage{zerosQ9[1]}i) \, )(x - \, (\sage{zerosQ9[0]}-\sage{zerosQ9[1]}i) \,)(\sage{hintValuesQ9[1]} x - \sage{hintValuesQ9[0]})$$

Now, we want to be careful how we multiply out.

$$(x^2 - \, (\sage{zerosQ9[0]}+\sage{zerosQ9[1]}i) \, x - \, (\sage{zerosQ9[0]}-\sage{zerosQ9[1]}i)\, x + (\sage{zerosQ9[0]}+\sage{zerosQ9[1]}i) (\sage{zerosQ9[0]}-\sage{zerosQ9[1]}i) \, ) \, (\sage{hintValuesQ9[1]} x - \sage{hintValuesQ9[0]})$$

$$(x^2 - \sage{2*zerosQ9[0]}x + (\sage{zerosQ9[0]}^2 + \sage{zerosQ9[1]}^2) \, )(\sage{hintValuesQ9[1]} x - \sage{hintValuesQ9[0]})$$

Distribute this last part out carefully and you will have completed the problem.
\end{hint}
\end{question}

\end{document}
