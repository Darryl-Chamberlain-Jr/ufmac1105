\documentclass{ximera}
\usepackage{sagetex}
\usepackage{multicol}
%% handout
%% space
%% newpage
%% numbers
%% nooutcomes

%% You can put user macros here
%% However, you cannot make new environments

\graphicspath{{./}{module1Activity/}{module2Activity/}{module3Activity/}}

\usepackage{sagetex}
\usepackage{tikz}
\usepackage{hyperref}
\usepackage{tkz-euclide}
\usetkzobj{all}
\pgfplotsset{compat=1.7} % prevents compile error.

\tikzstyle geometryDiagrams=[ultra thick,color=blue!50!black]
 %% we can turn off input when making a master document

\outcome{Understand and solve quadratic equations.}
\author{Darryl Chamberlain Jr.}
 
\title{Objective 4 - Solving Quadratic Equations with Quadratic Formula}

\begin{document}
\begin{abstract}
Solving quadratic equations using the Quadratic Formula.
\end{abstract}
\maketitle

\href{https://cnx.org/contents/mwjClAV_@8.1:-Sm9he1Q@17/Quadratic-Functions}{Link to section in online textbook.}

%%%%%%%%%%%%%%%%%%%%%
%%%  Objective 3  %%%
%%%%%%%%%%%%%%%%%%%%%

First, watch \underline{\href{https://mediasite.video.ufl.edu/Mediasite/Play/75f74a5b7d244e30b1b4e24253f595f61d}{this video}} to learn how to solve quadratic equations using the Quadratic Formula. Remember: this method will \textbf{always} work, but may not be the easiest method. 

Now practice solving quadratic equations with the quadratic formula. 

\begin{sagesilent}
x = var("x")

################################
def maybeMakeNegative(natural):
    integer = natural*(-1)**ZZ.random_element(2)
    return integer

def generateCoefficients():
    a = maybeMakeNegative(ZZ.random_element(2, 10))
    b = maybeMakeNegative(ZZ.random_element(2, 10))
    c = maybeMakeNegative(ZZ.random_element(2, 10))
    coefficients = [a, b, c]
    discrim = findDiscriminant(coefficients)
    while (discrim < 0 or discrim == 0 or isSquare(discrim)==true):
        a = maybeMakeNegative(ZZ.random_element(2, 10))
        b = maybeMakeNegative(ZZ.random_element(2, 10))
        c = maybeMakeNegative(ZZ.random_element(2, 10))
        coefficients = [a, b, c]
        discrim = findDiscriminant(coefficients)    
    return [a, b, c]

def generateSolution(coefficients):
    a, b, c = coefficients
    solveThisQuadratic = a*x**2+b*x+c
    solution = solve(solveThisQuadratic, x)
    if (solution[0].rhs() < solution[1].rhs()):
        x0 = round(float(solution[0].rhs()), 3)
        x1 = round(float(solution[1].rhs()), 3)
        answer = [x0, x1]
        return answer
    else:
        x0 = round(float(solution[1].rhs()), 3)
        x1 = round(float(solution[0].rhs()), 3)
        answer = [x0, x1]
        return answer

def findDiscriminant(coefficients):
    a, b, c = coefficients
    return b**2 - 4*a*c

def isSquare(integer):
    root = sqrt(integer)
    typeInteger = type(3)
    if type(root) == typeInteger:
        return True
    else:
        return False

################################
coefficients11 = generateCoefficients()
solution11 = generateSolution(coefficients11)
problemDisplay11 = coefficients11[0]*x**2 + coefficients11[1]*x + coefficients11[2]

coefficients12 = generateCoefficients()
solution12 = generateSolution(coefficients12)
problemDisplay12 = coefficients12[0]*x**2 + coefficients12[1]*x + coefficients12[2]

coefficients13 = generateCoefficients()
solution13 = generateSolution(coefficients13)
problemDisplay13 = coefficients13[0]*x**2 + coefficients13[1]*x + coefficients13[2]
\end{sagesilent}

\begin{question}
Solve the quadratic below.  

	$$ \sage{problemDisplay11} = 0 $$

Smaller solution: $x = \answer[tolerance=0.05]{\sage{solution11[0]}}$

Larger solution: $x = \answer[tolerance=0.05]{\sage{solution11[1]}}$	

\end{question}

\begin{question}
Solve the quadratic below. 

	$$ \sage{problemDisplay12} = 0 $$

Smaller solution: $x = \answer[tolerance=0.05]{\sage{solution12[0]}}$

Larger solution: $x = \answer[tolerance=0.05]{\sage{solution12[1]}}$	

\end{question}

\begin{question}
Solve the quadratic below. 

	$$ \sage{problemDisplay13} = 0 $$

Smaller solution: $x = \answer[tolerance=0.05]{\sage{solution13[0]}}$

Larger solution: $x = \answer[tolerance=0.05]{\sage{solution13[1]}}$	

\end{question}

\end{document}
