\documentclass{ximera}
\usepackage{sagetex}
%% handout
%% space
%% newpage
%% numbers
%% nooutcomes

%% You can put user macros here
%% However, you cannot make new environments

\graphicspath{{./}{module1Activity/}{module2Activity/}{module3Activity/}}

\usepackage{sagetex}
\usepackage{tikz}
\usepackage{hyperref}
\usepackage{tkz-euclide}
\usetkzobj{all}
\pgfplotsset{compat=1.7} % prevents compile error.

\tikzstyle geometryDiagrams=[ultra thick,color=blue!50!black]
 %% we can turn off input when making a master document

\outcome{Understand quadratic functions.}
\author{Darryl Chamberlain Jr.}
 
\title{Objective 1 - Factor trinomials}

\begin{document}
\begin{abstract}
Factor a trinomial with leading coefficient 1 and greater than 1. 
\end{abstract}
\maketitle

\href{https://cnx.org/contents/mwjClAV_@8.1:Sc8taR03@9/Factoring-Polynomials}{Link to section in online textbook}

One of the most important tools we will learn in this class is how to factor. We will need it in nearly all Modules from now on. 

Watch \underline{\href{https://mediasite.video.ufl.edu/Mediasite/Play/fba151da987447b9953fc738071bd4411d}{this video}} to review the basics for factoring trinomials. \textbf{Important: You may be able to factor these trinomials by guessing and checking. However, learning this basic technique will help you when you need to factor more difficult trinomials!}

Now try to factor the following polynomials. 

\begin{sagesilent}
x = var("x")
#################
def maybeMakeNegative(rational):
    maybeNegative = (-1)**ZZ.random_element(1)
    rational = maybeNegative * rational
    return rational
def generateNumber(maxRange):
    number = 2*(ZZ.random_element(maxRange-2)+2)
    return number
def createZeros(maxRange):
    c = maybeMakeNegative(generateNumber(maxRange))
    d = maybeMakeNegative(generateNumber(maxRange))
    while (c==-d or c==d):
        c = maybeMakeNegative(generateNumber(maxRange))
        d = maybeMakeNegative(generateNumber(maxRange))
    if (c < d or c==d):
        return [c, d]
    else:
        return [d, c]
##### END OF DEFINTIONS #####

###### QUESTION 4 #######
zeros4 = createZeros(14)
quadratic4 = x**2 + (zeros4[0] + zeros4[1]) * x + zeros4[0]*zeros4[1]
factor4A = x+zeros4[0]
factor4B = x+zeros4[1]
###### QUESTION 5 #######
zeros5 = createZeros(14)
quadratic5 = x**2 + (zeros5[0] + zeros5[1]) * x + zeros5[0]*zeros5[1]
factor5A = x+zeros5[0]
factor5B = x+zeros5[1]
\end{sagesilent}

\begin{question}

Factor the trinomial below. 

$\sage{quadratic4}$ 

($\answer{\sage{factor4A}}$) ($\answer{\sage{factor4B}}$)

\end{question}

\begin{question}

Factor the trinomial below. 
	
$\sage{quadratic5}$ 

($\answer{\sage{factor5A}}$)($\answer{\sage{factor5B}}$) 

\end{question}

Now that we have the basic technique to factor trinomials, we can focus on more difficult trinomials. Watch \underline{\href{https://mediasite.video.ufl.edu/Mediasite/Play/a022a3fb56194931b3f6c9833a9cd2251d}{this video}} to learn how to use the previous technique and extend it to trinomials with leading coefficient greater than 1.

Now try to factor the following polynomials. 

\begin{sagesilent}
x = var("x")
import random
import math

#######################
def maybeMakeNegative(rational):
    maybeNegative = (-1)**ZZ.random_element(1)
    rational = maybeNegative * rational
    return rational

def generateZeros(minimum, maximum):
    a = ZZ.random_element(maximum-minimum)+minimum
    b = maybeMakeNegative(ZZ.random_element(maximum-minimum)+minimum)
    while gcd(abs(a), abs(b)) > 1:
        a = ZZ.random_element(maximum-minimum)+minimum
        b = maybeMakeNegative(ZZ.random_element(maximum-minimum)+minimum)
    return [a, -b]

########## END OF DEFINITIONS #############

##### QUESTION 6 ######
zeros6 = generateZeros(3, 9)
quadratic6 = zeros6[0]**2 * x**2 - zeros6[1]**2
factor6A = zeros6[0]*x-abs(zeros6[1])
factor6B = zeros6[0]*x+abs(zeros6[1])

##### QUESTION 7 ######
zeros7 = generateZeros(3, 9)
quadratic7 = zeros7[0]**2 * x**2 - zeros7[1]**2
factor7A = zeros7[0]*x-abs(zeros7[1])
factor7B = zeros7[0]*x+abs(zeros7[1])

\end{sagesilent}

\begin{question}

Factor the polynomial below. 

$\sage{quadratic6}$

($\answer{\sage{factor6A}}$)($\answer{\sage{factor6B}}$)

\end{question}

\begin{question}
	Factor the polynomial below. 
	
	$\sage{quadratic7}$

	($\answer{\sage{factor7A}}$)($\answer{\sage{factor7B}}$)
\end{question}

%%%%%%%%%%%%%%%%%

\begin{sagesilent}
x = var("x")
import random
import math

##################
def maybeMakeNegative(rational):
    maybeNegative = (-1)**ZZ.random_element(1)
    rational = maybeNegative * rational
    return rational

def generateSolution(minimum, maximum):
    listPrimes = prime_range(minimum, maximum)
    a, c = random.sample(listPrimes, 2)
    #
    b = maybeMakeNegative(randint(minimum, maximum))
    d = maybeMakeNegative(randint(minimum, maximum))
    # This makes sure we can't factor out a constant.
    while (gcd(a, b)*gcd(c, d)>1):
        a, c = random.sample(listPrimes, 2)
        b = maybeMakeNegative(randint(minimum, maximum))
        d = maybeMakeNegative(randint(minimum, maximum))
    #This will guarantee that we always generate solutions with b < = d# 
    if(b < d or b == d):
        return [Integer(a), Integer(b), Integer(c), Integer(d)]
    else:
        return[Integer(c), Integer(d), Integer(a), Integer(b)]

def generateProblem(solution):
    a, b, c, d = solution
    coeffA = Integer(a*c)
    coeffB = Integer(a*d+b*c)
    coeffC = Integer(b*d)
    return [coeffA, coeffB, coeffC]

######### END OF DEFINITIONS #########
minimum = 3
maximum = 10

##### QUESTION 8 #####
coefficients8 = generateSolution(minimum, maximum)
a8, b8, c8 = generateProblem(coefficients8)
quadratic8 = a8 * x**2 + b8 * x + c8
factor8A = coefficients8[0]*x+coefficients8[1]
factor8B = coefficients8[2]*x+coefficients8[3]

##### QUESTION 9 #####
coefficients9 = generateSolution(minimum, maximum)
a9, b9, c9 = generateProblem(coefficients9)
quadratic9 = a9 * x**2 + b9 * x + c9
factor9A = coefficients9[0]*x+coefficients9[1]
factor9B = coefficients9[2]*x+coefficients9[3]
\end{sagesilent}

\begin{question}
Factor the polynomial below. 

$\sage{quadratic8}$

($\answer{\sage{factor8A}}$)($\answer{\sage{factor8B}}$)

\end{question}

\begin{question}
Factor the polynomial below. 

$\sage{quadratic9}$ 

($\answer{\sage{factor9A}}$)($\answer{\sage{factor9B}}$)

\end{question}

%%%%%%%%%%%%%%%%

\begin{sagesilent}
x = var("x")
import numpy
import random

##################
def maybeMakeNegative(rational):
    maybeNegative = (-1)**(randint(0, 1))
    rational = maybeNegative * rational
    return rational

def generateFactors(minimumPrime, maximumPrime, numberOfFactors):
    listPrimes = prime_range(minimum, maximum)
    aFactors = [random.sample(listPrimes, 1) for i in range(numberOfFactors)]
    cFactors = [random.sample(listPrimes, 1) for i in range(numberOfFactors)]
    return [aFactors, cFactors]

def generateSolution(minimum, maximum, factors):
    aFactors = factors[0]
    cFactors = factors[1]
    a = numpy.prod(aFactors)
    c = numpy.prod(cFactors)
    b = maybeMakeNegative(randint(minimum, maximum))
    d = maybeMakeNegative(randint(minimum, maximum))
    # This makes sure we can't factor out a constant.
    while ((gcd(a, b)*gcd(c, d)>1) or (a==c and b==-d)):
        aFactors = factors[0]
        cFactors = factors[1]
        a = numpy.prod(aFactors)
        c = numpy.prod(cFactors)
        b = maybeMakeNegative(randint(minimum, maximum))
        d = maybeMakeNegative(randint(minimum, maximum))
    #This will guarantee that we always generate solutions with b < = d
    if(b < d or b==d):
        return [Integer(a), Integer(b), Integer(c), Integer(d)]
    else:
        return[Integer(c), Integer(d), Integer(a), Integer(b)]

def generateProblem(solution):
    a, b, c, d = solution
    coeffA = Integer(a*c)
    coeffB = Integer(a*d+b*c)
    coeffC = Integer(b*d)
    return [coeffA, coeffB, coeffC]

######### END OF DEFINITIONS #########
minimum = 2
maximum = 7
numberOfFactors = 2

##### QUESTION 10 #####
factors10 = generateFactors(minimum, maximum, numberOfFactors)
solution10 = generateSolution(minimum, maximum, factors10)
coefficients10 = generateProblem(solution10)
quadratic10 = coefficients10[0] * x**2 + coefficients10[1] * x + coefficients10[2]
factor10A = solution10[0]*x + solution10[1]
factor10B = solution10[2]*x + solution10[3]

##### QUESTION 11 #####
factors11 = generateFactors(minimum, maximum, numberOfFactors)
solution11 = generateSolution(minimum, maximum, factors11)
coefficients11 = generateProblem(solution11)
quadratic11 = coefficients11[0] * x**2 + coefficients11[1] * x + coefficients11[2]
factor11A = solution11[0]*x + solution11[1]
factor11B = solution11[2]*x + solution11[3]

##### QUESTION 12 #####
factors12 = generateFactors(minimum, maximum, numberOfFactors)
solution12 = generateSolution(minimum, maximum, factors12)
coefficients12 = generateProblem(solution12)
quadratic12 = coefficients12[0] * x**2 + coefficients12[1] * x + coefficients12[2]
factor12A = solution12[0]*x + solution12[1]
factor12B = solution12[2]*x + solution12[3]
\end{sagesilent}

\begin{question}
Factor the polynomial below. 

$\sage{quadratic10}$

($\answer{\sage{factor10A}}$)($\answer{\sage{factor10B}}$)

\end{question}

\begin{question}
	Factor the polynomial below. 
	
	$\sage{quadratic11}$

	($\answer{\sage{factor11A}}$)($\answer{\sage{factor11B}}$)

\end{question}

\begin{question}
	Factor the polynomial below. 
	
	$\sage{quadratic12}$

	($\answer{\sage{factor12A}}$)($\answer{\sage{factor12B}}$)

\end{question}

\end{document}
