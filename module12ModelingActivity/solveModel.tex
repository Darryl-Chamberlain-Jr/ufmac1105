\documentclass{ximera}
\usepackage{sagetex}
%% handout
%% space
%% newpage
%% numbers
%% nooutcomes
 
%% You can put user macros here
%% However, you cannot make new environments

\graphicspath{{./}{module1Activity/}{module2Activity/}{module3Activity/}}

\usepackage{sagetex}
\usepackage{tikz}
\usepackage{hyperref}
\usepackage{tkz-euclide}
\usetkzobj{all}
\pgfplotsset{compat=1.7} % prevents compile error.

\tikzstyle geometryDiagrams=[ultra thick,color=blue!50!black]
 %% we can turn off input when making a master document
 
\outcome{}
\author{Darryl Chamberlain Jr.}
  
\title{Objective 3 - Solving Real-World Problems}
 
\begin{document}
\begin{abstract}

\end{abstract}

\maketitle
 
\textit{Note: There are no textbook or videos directly to this section. If you want to review a certain type of model, you will need to go back to that Module.}
 
%%%%%%%%%%%%%%%%%%%%%
%%%  Objective 3  %%%
%%%%%%%%%%%%%%%%%%%%%

We have spent a lot of time building models. Once we have a model, solving a real-world problem is just a matter of plugging in some values into your model. 

% LINEAR
\begin{sagesilent}
v = var('v')
concA = ZZ.random_element(1, 4)*5
concB = ZZ.random_element(2, 5)*10
concTotal = ZZ.random_element(10, 31)
while concTotal < concA or concTotal > concB:
    concTotal = ZZ.random_element(10, 31)
totalVolume = ZZ.random_element(5, 15)
eqPartA6 = totalVolume - v
eqPartB6 = concA * 0.01 * v
eqPartC6 = concB * 0.01 * eqPartA6
eqPartD6 = eqPartB6 + eqPartC6
volumeA = solve(eqPartD6==concTotal*totalVolume*0.01, v)[0].rhs()
volumeB = eqPartA6(volumeA)
\end{sagesilent}
\begin{question}
Chemists commonly create a solution by mixing two products of differing concentrations together.  Find the amount of $\sage{concA}\%$ acid solution and $\sage{concB}\%$ acid solution needed to create a $\sage{totalVolume}$ liter $\sage{concTotal}$\% solution. 

$\sage{concA}\%$ acid solution: $\answer[tolerance=0.05]{\sage{volumeA}}$ liters

$\sage{concB}\%$ acid solution: $\answer[tolerance=0.05]{\sage{volumeB}}$ liters 

\end{question}

% EXPONENTIAL 
\begin{sagesilent}
t = var('t')
halfLifeYears2 = ZZ.random_element(10, 101)*ZZ.random_element(10, 101)
afterYears = halfLifeYears2 + ZZ.random_element(50, 101)
initialAmount2 = ZZ.random_element(100, 1000)
k2 = ln(1/2)/halfLifeYears2
equation2 = initialAmount2*e**(k2*t)
amountLeft = equation2(afterYears)
\end{sagesilent}
\begin{question}
There is initially $\sage{initialAmount2}$ grams of element $X$. The half-life of element $X$ is $\sage{halfLifeYears2}$ years. How much element $X$ will be left after $\sage{afterYears}$ years?

$\answer[tolerance=0.01]{\sage{amountLeft}}$ grams
\end{question}

% LINEAR
\begin{sagesilent}
x = var("x")
fixedCost1 = ZZ.random_element(10, 26)*1000
productionCost1 = round(ZZ.random_element(1, 4)*0.05, 2)
sellingPoint1 = round(productionCost1*ZZ.random_element(2, 5), 2)
costs1 = productionCost1*x + fixedCost1
profits1 = sellingPoint1*x
revenue1 = profits1 - costs1
exactBreakEven = solve(revenue1==0, x)[0].rhs()
if exactBreakEven > round(exactBreakEven, 0):
    breakEven = round(exactBreakEven, 0) + 1
else:
    breakEven = round(exactBreakEven, 0) 
\end{sagesilent}

\begin{question}
A company sells doughnuts. They incur a fixed cost of \$$\sage{fixedCost1}$ for rent, insurance, and other expenses. It costs \$$\sage{productionCost1}$ to produce each doughnut. The company sells each doughnut for \$$\sage{sellingPoint1}$. How many doughnuts would they need to sell to break even?

$\answer{\sage{breakEven}}$ doughnuts

\end{question}

% DIRECT 
\begin{sagesilent}
k1 = (4*pi**2) / (9.8)
a1 = ZZ.random_element(3, 9)
saturnA = a1 - ZZ.random_element(1, 3)
while (saturnA > a1):
    a1 = ZZ.random_element(3, 9)
    saturnA = a1 - ZZ.random_element(1, 3)
T1 = round(k1*sqrt(a1**3)*12, 2)
saturnMonths = round(k1*sqrt(saturnA**3)*12, 2)

\end{sagesilent}

\begin{question}
Kepler's Third Law: The square of the time, $T$, required for a planet to orbit the Sun is directly proportional to the cube of the mean distance, $a$, that the planet is from the Sun. Assume that Neptune's mean distance from the Sun is $\sage{a1}$ AUs and it takes Neptune about $\sage{T1}$ months to orbit the Sun. If it takes Saturn about $\sage{saturnMonths}$ months to orbit the Sun, what is Saturn's mean distance from the Sun?

$\answer[tolerance=0.05]{\sage{saturnA}}$

\end{question}

\begin{sagesilent}
t = var('t')
r = var('r')
k3 = ln(0.5)/5730
equation3A = e**(t*k3)
equation3B = ln(r)/k3
percent3 = ZZ.random_element(1, 101)
old3 = round(equation3B(percent3*0.01), 0)
\end{sagesilent}

\begin{question}
The half-life of carbon-14 is 5,730 years. A bone fragment is found that contains $\sage{percent3}\%$ of its original carbon-14. To the nearest year, how old is the bone?

$\answer{\sage{old3}} \, \text{ years old}$

\end{question}

% LINEAR
\begin{sagesilent}
m = var('m')
officerAspeed3 = ZZ.random_element(2, 7)
officerBspeed3 = ZZ.random_element(2, 7)
while officerAspeed3 == officerBspeed3:
    officerAspeed3 = ZZ.random_element(2, 7)
    officerBspeed3 = ZZ.random_element(2, 7)
partA3 = officerAspeed3/60 * m
partB3 = officerAspeed3/60 * m + officerBspeed3/60 * m
partC3 = sqrt((officerAspeed3/60)**2 + (officerBspeed3/60)**2)*m
distanceOfficers = ZZ.random_element(2, 5)
timeOfficers = solve(partC3 == distanceOfficers, m)[0].rhs()
\end{sagesilent}

\begin{question}
Two UFPD are patrolling the campus on foot. To cover more ground, they split up and begin walking in different directions. Office A is walking at $\sage{officerAspeed3}$ mph directly south while Office B is walking at $\sage{officerBspeed3}$ mph directly west. How long would they need to walk before they are $\sage{distanceOfficers}$ miles away from each other?

$\answer[tolerance=0.05]{\sage{timeOfficers}}$ minutes

\end{question}

\begin{sagesilent}
t = var('t')
rateList1 = ["doubles", "triples", "quadruples"]
choose1 = ZZ.random_element(0, 3)
rate1 = rateList1[choose1]
initialBacteria1 = ZZ.random_element(1, 9)*100
bacteriaPopulation1 = initialBacteria1 * (choose1 + 2)**(t)
hugeBacteriaPopulation = ZZ.random_element(1, 9)
timeToHugePopulation = solve(bacteriaPopulation1 == hugeBacteriaPopulation*1000000, t)[0].rhs()
\end{sagesilent}
\begin{question}
A population of bacteria $\sage{rate1}$ every hour. If the culture started with $\sage{initialBacteria1}$, how long would it take before the population is over $\sage{hugeBacteriaPopulation}$ million?

$\answer[tolerance=0.05]{\sage{timeToHugePopulation}}$ hours

\end{question}

\end{document}