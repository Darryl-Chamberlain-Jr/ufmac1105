\documentclass{ximera}
\usepackage{sagetex}
%% handout
%% space
%% newpage
%% numbers
%% nooutcomes

%% You can put user macros here
%% However, you cannot make new environments

\graphicspath{{./}{module1Activity/}{module2Activity/}{module3Activity/}}

\usepackage{sagetex}
\usepackage{tikz}
\usepackage{hyperref}
\usepackage{tkz-euclide}
\usetkzobj{all}
\pgfplotsset{compat=1.7} % prevents compile error.

\tikzstyle geometryDiagrams=[ultra thick,color=blue!50!black]
 %% we can turn off input when making a master document

\outcome{}
\author{Darryl Chamberlain Jr.}
 
\title{Objective 3 - Solving Rational Equations}

\begin{document}
\begin{abstract}
Solve rational equations that lead to linear and quadratic equations.
\end{abstract}
\maketitle

\href{https://cnx.org/contents/mwjClAV_@8.1:KNTP2r7D@13/Rational-Functions}{Link to section in online textbook.}

%%%%%%%%%%%%%%%%%%%%%
%%%  Objective 4  %%%
%%%%%%%%%%%%%%%%%%%%%

You can print out \href{http://people.clas.ufl.edu/dchamberlain31/files/M7-Objective-3-Solve-Rational-Equalities.pdf}{these notes} to follow along with the video below and keep notes to organize your thoughts.

\youtube{D2nROy65AOI}

\begin{sagesilent}
x = var("x")

def maybeMakeNegative(natural):
    integer = natural*(-1)**ZZ.random_element(2)
    return integer 

def generateTermsLinearSolutions(numberOfSolutions):
    if (numberOfSolutions==0):
        mask = ZZ.random_element(3, 7)
        a = maybeMakeNegative(ZZ.random_element(2, 9)) * mask
        d = -1
        f = 1
        e = a
        b = maybeMakeNegative(ZZ.random_element(2, 9)) * mask
        c = maybeMakeNegative(ZZ.random_element(2, 9)) * mask
        solution = 0
    else:
        a = maybeMakeNegative(ZZ.random_element(2, 9))
        b = maybeMakeNegative(ZZ.random_element(2, 9))
        c = maybeMakeNegative(ZZ.random_element(2, 9))
        d = maybeMakeNegative(ZZ.random_element(2, 9))
        e = maybeMakeNegative(ZZ.random_element(2, 9))
        f = maybeMakeNegative(ZZ.random_element(2, 9))
        makeSolutionExist = (-f*a +e + d*c*f)/(-d*f*b)
        #
        while (makeSolutionExist==0):
            a = maybeMakeNegative(ZZ.random_element(2, 9))
            b = maybeMakeNegative(ZZ.random_element(2, 9))
            c = maybeMakeNegative(ZZ.random_element(2, 9))
            d = maybeMakeNegative(ZZ.random_element(2, 9))
            e = maybeMakeNegative(ZZ.random_element(2, 9))
            f = maybeMakeNegative(ZZ.random_element(2, 9))
            makeSolutionExist = (-f*a +e + d*c*f)/(-d*f*b)
        solution = (-f*a +e + d*c*f)/(-d*f*b)
    leftTerm = (a/(b*x+c)) - d
    rightTermNum = e
    rightTermDenom = f*(b*x+c)
    return [leftTerm, rightTermNum, rightTermDenom, solution]

def generateCoefficientsQuadraticSolutions(numberOfSolutions):
    #
    a = maybeMakeNegative(ZZ.random_element(2, 7))
    b = maybeMakeNegative(ZZ.random_element(2, 7))
    c = maybeMakeNegative(ZZ.random_element(2, 7))
    d = -ZZ.random_element(2, 7)
    e = maybeMakeNegative(ZZ.random_element(2, 7))
    f = maybeMakeNegative(ZZ.random_element(2, 7))
    g = maybeMakeNegative(ZZ.random_element(2, 7))
    discrim = (a*g-e*b)**2 - 4*(a*f+d)*(-e*c)
    leadingCoefficient = a*f+d

   #
    if (numberOfSolutions == 0):
        while (leadingCoefficient == 0 or discrim > 0 or discrim == 0):
            a = maybeMakeNegative(ZZ.random_element(2, 7))
            b = maybeMakeNegative(ZZ.random_element(2, 7))
            c = maybeMakeNegative(ZZ.random_element(2, 7))
            d = -ZZ.random_element(2, 7)
            e = maybeMakeNegative(ZZ.random_element(2, 7))
            f = maybeMakeNegative(ZZ.random_element(2, 7))
            g = maybeMakeNegative(ZZ.random_element(2, 7))
            discrim = (a*g-e*b)**2 - 4*(a*f+d)*(-e*c)
            leadingCoefficient = a*f+d
        solution = [0, 0]
    elif (numberOfSolutions == 1):
        while (leadingCoefficient==0 or discrim != 0):
            a = maybeMakeNegative(ZZ.random_element(2, 7))
            b = maybeMakeNegative(ZZ.random_element(2, 7))
            c = maybeMakeNegative(ZZ.random_element(2, 7))
            d = -ZZ.random_element(2, 7)
            e = maybeMakeNegative(ZZ.random_element(2, 7))
            f = maybeMakeNegative(ZZ.random_element(2, 7))
            g = maybeMakeNegative(ZZ.random_element(2, 7))
            discrim = (a*g-e*b)**2 - 4*(a*f+d)*(-e*c)
            leadingCoefficient = a*f+d
        quadraticEquation = (a*x)/(b*x+c) + (d*x**2)/(b*f*x**2 + (b*g+c*f)*x+c*g) - e/(f*x+g)
        singleSolution = solve(quadraticEquation == 0, x)[0].rhs()
        solution = [singleSolution, 0]
    else:
        while (leadingCoefficient==0 or discrim < 0 or discrim == 0):
            a = maybeMakeNegative(ZZ.random_element(2, 7))
            b = maybeMakeNegative(ZZ.random_element(2, 7))
            c = maybeMakeNegative(ZZ.random_element(2, 7))
            d = -ZZ.random_element(2, 7)
            e = maybeMakeNegative(ZZ.random_element(2, 7))
            f = maybeMakeNegative(ZZ.random_element(2, 7))
            g = maybeMakeNegative(ZZ.random_element(2, 7))
            discrim = (a*g-e*b)**2 - 4*(a*f+d)*(-e*c)
            leadingCoefficient = a*f+d
        quadraticEquation = (a*x)/(b*x+c) + (d*x**2)/(b*f*x**2 + (b*g+c*f)*x+c*g) - e/(f*x+g)
        rawSolutions = solve(quadraticEquation == 0, x)
        solution1 = rawSolutions[0].rhs()
        solution2 = rawSolutions[1].rhs()
        solution = sorted([solution1, solution2])
    firstTerm = (a*x)/(b*x+c)
    secondTerm = (d*x**2)/(b*f*x**2 + (b*g+c*f)*x+c*g)
    thirdTerm = e/(f*x+g)
    return [firstTerm, secondTerm, thirdTerm, solution]
##########
leftTerm1, rightTerm1Num, rightTerm1Denom, solution1 = generateTermsLinearSolutions(0)
leftTerm2, rightTerm2Num, rightTerm2Denom, solution2 = generateTermsLinearSolutions(1)
#
firstTerm3, secondTerm3, thirdTerm3, solution3TEMP = generateCoefficientsQuadraticSolutions(0)
leftTerm3 = firstTerm3 + secondTerm3
#
firstTerm4, secondTerm4, thirdTerm4, solution4TEMP = generateCoefficientsQuadraticSolutions(1)
leftTerm4 = firstTerm4 + secondTerm4
solution4 = solution4TEMP[0]
#
firstTerm5, secondTerm5, thirdTerm5, solution5TEMP = generateCoefficientsQuadraticSolutions(2)
leftTerm5 = firstTerm5 + secondTerm5
solution5a, solution5b = sorted([solution5TEMP[0], solution5TEMP[1]])
\end{sagesilent}

% Q1 - Linear, 0 solutions
\begin{question}
Solve the rational equation below. \textit{Remember to check your solutions to make sure they are valid!} If there is no solution, answer ``NA".

$ \sage{leftTerm1} = \frac{\sage{rightTerm1Num}}{\sage{rightTerm1Denom}}$

Solution: $x = \answer[format=string]{NA}$

% Remember to give tolerances when necessary
\end{question}

% Q2 - Linear, 1 solution
\begin{question}
Solve the rational equation below. \textit{Remember to check your solutions to make sure they are valid!} If there is no solution, answer ``NA".

$ \sage{leftTerm2} = \frac{\sage{rightTerm2Num}}{\sage{rightTerm2Denom}}$

Solution: $x = \answer[tolerance=0.05]{\sage{solution2}}$

% Remember to give tolerances when necessary
\end{question}

% Q3 - Quadratic, 0 solutions
\begin{question}
Solve the rational equation below. \textit{Remember to check your solutions to make sure they are valid!} If there are more boxes than solutions, answer ``NA".

$ \sage{leftTerm3} = \sage{thirdTerm3}$

Solutions: $x = \answer[format=string]{NA}$ and $x = \answer[format=string]{NA}$

% Remember to give tolerances when necessary
\end{question}

% Q4 - Quadratic, 1 solution
\begin{question}
Solve the rational equation below. \textit{Remember to check your solutions to make sure they are valid!} If there are more boxes than solutions, answer ``NA".

$ \sage{leftTerm4} = \sage{thirdTerm4}$

Solutions: $x = \answer[tolerance=0.05]{\sage{solution4}}$ and $x = \answer[format=string]{NA}$

% Remember to give tolerances when necessary
\end{question}

% Q5 - Quadratic, 2 solutions
\begin{question}
Solve the rational equation below. \textit{Remember to check your solutions to make sure they are valid!} If there are more boxes than solutions, answer ``NA".

$ \sage{leftTerm5} = \sage{thirdTerm5}$

Solutions: $x = \answer[tolerance=0.05]{\sage{solution5a}}$ and $x = \answer[tolerance=0.05]{\sage{solution5b}}$
% Remember to give tolerances when necessary
\end{question}

% TAKEAWAY
\begin{question}
\textbf{Main takeaway:} Before looking, you should work through the previous problems. \textit{Have you finished working through the examples?} $\answer[format=string]{Yes}$
\begin{feedback}[correct]
To solve rational equations, we want to multiply to remove the denominators. When in doubt, multiply by the denominator of each one at a time. This may not always be the most efficient way \textit{(multiplying by the GCD would be)} it will eventually get the equation into a more manageable form. Like with radical functions, we also need to check our solutions to make sure they are valid -- that we are not dividing by 0.
\end{feedback}
\end{question}

\end{document}