\documentclass{ximera}
\usepackage{sagetex}
%% handout
%% space
%% newpage
%% numbers
%% nooutcomes

%% You can put user macros here
%% However, you cannot make new environments

\graphicspath{{./}{module1Activity/}{module2Activity/}{module3Activity/}}

\usepackage{sagetex}
\usepackage{tikz}
\usepackage{hyperref}
\usepackage{tkz-euclide}
\usetkzobj{all}
\pgfplotsset{compat=1.7} % prevents compile error.

\tikzstyle geometryDiagrams=[ultra thick,color=blue!50!black]
 %% we can turn off input when making a master document

\outcome{}
\author{Darryl Chamberlain Jr.}
 
\title{Module 07 - Rational Functions}

\begin{document}
\begin{abstract}
Graphing rational functions and solving rational equations. 
\end{abstract}
\maketitle

Now that we've worked through polynomial functions and radical functions, we look at another class of functions: rational. These are functions where our variable is in the denominator. Like the radical functions, we'll need to take a closer look at the domains of these functions. 

After considering what the graphs of these functions look like, we will focus on solving rational equations. Similar to the radical equation section, the restricted domain of rational functions means we need to check that our solution actually works for the equation!

The objectives for this homework are: 
\begin{enumerate}
	\item \href{https://cnx.org/contents/mwjClAV_@8.1:KNTP2r7D@13/Rational-Functions}{Identify the domain of a rational function.}
    \item \href{https://cnx.org/contents/mwjClAV_@8.1:KNTP2r7D@13/Rational-Functions}{Convert between basic rational functions and their graphs.}
    \item \href{https://cnx.org/contents/mwjClAV_@8.1:KNTP2r7D@13/Rational-Functions}{Solve rational equations that lead to linear and quadratic equations.}
\end{enumerate}

\end{document}