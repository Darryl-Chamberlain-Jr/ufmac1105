\documentclass{ximera}
%\usepackage{sagetex}
%% handout
%% space
%% newpage
%% numbers
%% nooutcomes

%%% You can put user macros here
%% However, you cannot make new environments

\graphicspath{{./}{module1Activity/}{module2Activity/}{module3Activity/}}

\usepackage{sagetex}
\usepackage{tikz}
\usepackage{hyperref}
\usepackage{tkz-euclide}
\usetkzobj{all}
\pgfplotsset{compat=1.7} % prevents compile error.

\tikzstyle geometryDiagrams=[ultra thick,color=blue!50!black]
 %% we can turn off input when making a master document

\outcome{}
\author{Darryl Chamberlain Jr.}
 
\title{Objectives}

\begin{document}
\begin{abstract}
List of objectives for Module 9M - Modeling with Linear Functions.
\end{abstract}
\maketitle

% Introduction to the section. 
Models allow us to use mathematical functions to describe real-world phenomena. This could be as straightforward as using your previous months of bills to predict what your future bills will be (assuming it will stay relatively the same) or as volatile as using multiple polls to try to predict a presidential election winner. In this course, we'll focus on models that can be described with linear, quadratic, logarithmic, or exponential functions. 

% How does it relate to previous section?
Our first Module will focus on Modeling with Linear functions. 
% How will it be useful for the future?

The objectives for this Module are: 
\begin{enumerate}
% \href{url}{test}
    \item \href{https://cnx.org/contents/mwjClAV_@8.12:3PeE3KzR@10/Modeling-with-Linear-Functions}{Identify when a real-world situation would require a linear function.}
	\item Describe the domain on which the model is valid. \textit{There is no section in the textbook related to this objective.}
	\item \href{https://cnx.org/contents/mwjClAV_@8.12:6dX4RGdg@12/Fitting-Linear-Models-to-Data}{Construct a model equation for the real-life situation.} 
\end{enumerate}

\end{document}