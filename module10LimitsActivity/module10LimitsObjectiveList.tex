\documentclass{ximera}
%\usepackage{sagetex}
%% handout
%% space
%% newpage
%% numbers
%% nooutcomes

%%% You can put user macros here
%% However, you cannot make new environments

\graphicspath{{./}{module1Activity/}{module2Activity/}{module3Activity/}}

\usepackage{sagetex}
\usepackage{tikz}
\usepackage{hyperref}
\usepackage{tkz-euclide}
\usetkzobj{all}
\pgfplotsset{compat=1.7} % prevents compile error.

\tikzstyle geometryDiagrams=[ultra thick,color=blue!50!black]
 %% we can turn off input when making a master document

\outcome{}
\author{Darryl Chamberlain Jr.}
 
\title{Objectives}

\begin{document}
\begin{abstract}
List of objectives for Module 10L - Synthetic Division.
\end{abstract}
\maketitle

In Module 6, we looked at polynomials in general - both how they are graphed and how to build them. For the questions that asked you to think about the general shape (end behavior and zero behavior) the polynomials were in \textit{factored form} $a(x-z_1)^{n_1}(x-z_2)^{n_2} \cdots (x-z_k)^{n_k}$. In this form, we could easily see what the zeros were and could graph the end and zero behaviors. This Module will teach you how to go from the \textit{general form} $a_n x^n + a_{n-1} x^{n-1} + \cdots + a_1 x + a_0$ to the factored form by introducing a new way to factor: Synthetic Division.
 
The objectives for this homework are:
\begin{enumerate}
    \item \href{https://cnx.org/contents/mwjClAV_@8.1:EfK9BY7_@13/Dividing-Polynomials}{Divide two polynomials using Synthetic Division.}
    \item \href{https://cnx.org/contents/mwjClAV_@8.1:G7V9LHif@17/Zeros-of-Polynomial-Functions}{Determine the possible rational roots of a polynomial.}
    \item \href{https://cnx.org/contents/mwjClAV_@8.1:G7V9LHif@17/Zeros-of-Polynomial-Functions}{Use Synthetic Division to factor a polynomial completely.}
\end{enumerate}

\end{document}