\documentclass{ximera}
\usepackage{sagetex}
%% handout
%% space
%% newpage
%% numbers
%% nooutcomes
 
%% You can put user macros here
%% However, you cannot make new environments

\graphicspath{{./}{module1Activity/}{module2Activity/}{module3Activity/}}

\usepackage{sagetex}
\usepackage{tikz}
\usepackage{hyperref}
\usepackage{tkz-euclide}
\usetkzobj{all}
\pgfplotsset{compat=1.7} % prevents compile error.

\tikzstyle geometryDiagrams=[ultra thick,color=blue!50!black]
 %% we can turn off input when making a master document
 
\outcome{}
\author{Darryl Chamberlain Jr.}
  
\title{Objective 2 - Possible Rational Roots}
 
\begin{document}
\begin{abstract}
Determine the possible rational roots of a polynomial.
\end{abstract}
\maketitle
 
\href{https://cnx.org/contents/mwjClAV_@8.1:G7V9LHif@17/Zeros-of-Polynomial-Functions}{Link to section in online textbook.}
 
%%%%%%%%%%%%%%%%%%%%%
%%%  Objective 2  %%%
%%%%%%%%%%%%%%%%%%%%%
 
First, watch
\underline{\href{https://mediasite.video.ufl.edu/Mediasite/Play/53ddb4ecf01e431ca5590036addef90b1d}{this video}} to learn about the rational root theorem.
 
\begin{theorem}
Rational Root Theorem: The \underline{possible} rational roots of the polynomial $a_n x^n + a_{n-1} x^{n-1} + \cdots a_1 x + a_0$ are of the form $\pm \frac{p}{q}$, where $p$ is a divisor of $a_0$ and $q$ is a divisor of $a_n$.
\end{theorem}
 
\begin{question}
This question will walk you through how to list the possible rational roots for a given polynomial.
 
$ f(x) = 6x^3 - 17x^2 + 9x + 8 $
 
First, we identify $a_0$ and $a_n$:
 
$ a_0 = \answer{8} $
 
$ a_n = \answer{6} $
 
Next, we list the divisors of each number. We'll list them smallest to largest.
 
Divisors of $a_0: \answer{1}, \answer{2}, \answer{4}, \answer{8}$
\begin{feedback}
Don't forget 1 and the number itself!
\end{feedback}
 
Divisors of $a_n: \answer{1}, \answer{2}, \answer{3}, \answer{6}$
 
Now, we list every possible combination. \textit{This example illustrates a large number of possible rational roots.}
 
$\pm \frac{1}{1}, \pm \frac{1}{2}, \pm \frac{1}{3}, \pm \frac{1}{6}, \\ \\
\pm \frac{2}{1}, \pm \frac{2}{2}, \pm \frac{2}{3}, \pm \frac{2}{6}, \\ \\
\pm \frac{4}{1}, \pm \frac{4}{2}, \pm \frac{4}{3}, \pm \frac{4}{6}, \\ \\
\pm \frac{8}{1}, \pm \frac{8}{2}, \pm \frac{8}{3}, \pm \frac{8}{6}$
 
This may look like a lot (and it is!) but we also have a lot of copies. Fill in the reduced forms for each combination above.
 
$\pm \answer{1}, \pm \answer{1/2}, \pm \answer{1/3}, \pm \answer{1/6}, \\ \\
\pm \answer{2}, \pm \answer{1}, \pm \answer{2/3}, \pm \answer{1/3}, \\ \\
\pm \answer{4}, \pm \answer{2}, \pm \answer{4/3}, \pm \answer{2/3}, \\ \\
\pm \answer{8}, \pm \answer{4}, \pm \answer{8/3}, \pm \answer{4/3}$
 
Our final answer is listing each number once:
 
$ \pm 1, \pm \frac{1}{2}, \pm \frac{1}{3}, \pm \frac{1}{6}, 2, \pm \frac{2}{3}, \pm 4, \pm \frac{4}{3}, \pm 8, \pm \frac{8}{3} $
 
\textbf{It is important to keep in mind this is a list of all \underline{possible rational} roots. It is usually far more than the actual number of roots of the polynomial. It could even miss some of our roots as roots can be irrational! We will illustrate that issue with a few examples below.}
\end{question}
 
\begin{sagesilent}
listSomePrimes = [2, 3, 5, 7, 11]
divisor0, divisor1 = sample(listSomePrimes, 2)
notPerfectSquare = divisor0 * divisor1
if divisor1 > divisor0:
    smallerDivisor = divisor0
    largerDivisor = divisor1
else:
    smallerDivisor = divisor1
    largerDivisor = divisor0
solution1a = float(-sqrt(notPerfectSquare))
solution1b = float(sqrt(notPerfectSquare))
\end{sagesilent}
 
\begin{question}
List the possible rational roots of the polynomial below. Then, find the actual rational roots by factoring or using the Quadratic Formula.
 
$f(x) = x^2 - \sage{notPerfectSquare}$
 
Possible rational roots: $\pm \answer{1}, \pm \answer{\sage{smallerDivisor}}, \pm \answer{\sage{largerDivisor}}, \pm \answer{\sage{notPerfectSquare}}$
 
\begin{feedback}
List possible roots in order from smallest to largest.
\end{feedback}
 
Smaller root: $\answer[tolerance=0.05]{\sage{solution1a}}$
 
Larger root: $\answer[tolerance=0.05]{\sage{solution1b}}$
 
\begin{feedback}[correct]
Here's an example where the rational root theorem doesn't help us. Both of the zeros of this polynomial are irrational!
\end{feedback}
 
\end{question}
 
\begin{sagesilent}
listPerfectSquares = [4, 9, 25, 49, 121]
perfectSquare = sample(listPerfectSquares, 1)[0]
otherDivisor = sqrt(perfectSquare)
smallerRoot = -sqrt(perfectSquare)*i
largerRoot = sqrt(perfectSquare)*i
\end{sagesilent}
 
\begin{question}
List the possible rational roots of the polynomial below. Then, find the actual rational roots by factoring or using the Quadratic Formula.
 
$f(x) = x^2 + \sage{perfectSquare}$
 
Possible rational roots: $\pm \answer{1}, \pm \answer{\sage{otherDivisor}}, \pm \answer{\sage{perfectSquare}}$
 
\begin{feedback}
List possible roots in order from smallest to largest.
\end{feedback}
 
Smaller root: $\answer{\sage{smallerRoot}}$
 
Larger root: $\answer{\sage{largerRoot}}$
 
\begin{feedback}[correct]
Here's another example where the rational root theorem doesn't help us. Both of the zeros of this polynomial are complex!
\end{feedback}
 
\end{question}
 
In the next section, you will get plenty of practice finding the possible rational roots. Let's get to it.
 
\end{document}