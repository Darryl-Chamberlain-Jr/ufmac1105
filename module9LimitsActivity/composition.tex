\documentclass{ximera}
\usepackage{sagetex}
%% handout
%% space
%% newpage
%% numbers
%% nooutcomes
 
%% You can put user macros here
%% However, you cannot make new environments

\graphicspath{{./}{module1Activity/}{module2Activity/}{module3Activity/}}

\usepackage{sagetex}
\usepackage{tikz}
\usepackage{hyperref}
\usepackage{tkz-euclide}
\usetkzobj{all}
\pgfplotsset{compat=1.7} % prevents compile error.

\tikzstyle geometryDiagrams=[ultra thick,color=blue!50!black]
 %% we can turn off input when making a master document
 
\outcome{}
\author{Darryl Chamberlain Jr.}
  
\title{Objective 2 - Composition}
 
\begin{document}
\begin{abstract}
Evaluate the composition of two functions.
\end{abstract}
\maketitle
 
\href{https://cnx.org/contents/mwjClAV_@8.1:ik_Ed0Pa@12/Composition-of-Functions}{Link to section in online textbook.}
 
%%%%%%%%%%%%%%%%%%%%%
%%%  Objective 2  %%%
%%%%%%%%%%%%%%%%%%%%%
 
First, watch
% UPDATE VIDEO LINK
\underline{\href{https://mediasite.video.ufl.edu/Mediasite/Play/cf54ade8e8594719ac48e17841a745401d}{this video}} to learn how use a new operation on functions: Composition.
 
\begin{sagesilent}
x = var("x")
 
def maybeMakeNegative(natural):
    integer = natural*(-1)**ZZ.random_element(2)
    return integer
 
def evaluateNegativeCubeRoot(value, a, b):
    if value - b < 0:
        return -((-value+b)**(1/3)) + a
    else:
        return (value-b)**(1/3) + a
 
def generatePolynomial(degree):
    coefficients = [maybeMakeNegative(ZZ.random_element(3, 7)), maybeMakeNegative(ZZ.random_element(3, 7)), maybeMakeNegative(ZZ.random_element(3, 7)), maybeMakeNegative(ZZ.random_element(3, 7)), maybeMakeNegative(ZZ.random_element(3, 7))]
    if degree == 1:
        coefficients[4] = 0
        coefficients[3] = 0
        coefficients[2] = 0
    elif degree == 2:
        coefficients[4] = 0
        coefficients[3] = 0
    else:
        coefficients[4] = 0
    polynomial = coefficients[4] * x**4 + coefficients[3] * x**3 + coefficients[2] * x**2 + coefficients[1] * x + coefficients[0]
    return polynomial
 
def generateRadical():
    factor = generatePolynomial(1)
    pivotPoint = round(float(solve(factor == 0, x)[0].rhs() ), 3)
    checkNearby = pivotPoint - 1
    if factor(x=checkNearby) < 0:
        domain = [pivotPoint, Infinity]
    else:
        domain = [-Infinity, pivotPoint]
    radical = sqrt(factor)
    return [radical, domain]
 
def generateRational():
    factor = generatePolynomial(1)
    excludeFromDomain = round(float(solve(factor == 0, x)[0].rhs() ), 3)
    rational = 1/factor
    return [rational, excludeFromDomain]
 
def generateInversePair():
    a = maybeMakeNegative(ZZ.random_element(3, 7))
    b = maybeMakeNegative(ZZ.random_element(3, 7))
    while a == b or a == -b:
        a = maybeMakeNegative(ZZ.random_element(3, 7))
        b = maybeMakeNegative(ZZ.random_element(3, 7))
    poly = (x - a)**3 + b
    radical = (x-b)**(1/3) + a
    return [poly, radical, a, b]
#####
degreeQ1f = ZZ.random_element(1, 3)
degreeQ1g = ZZ.random_element(1, 3)
functionQ1f = generatePolynomial(degreeQ1f)
functionQ1g = generatePolynomial(degreeQ1g)
valueQ1 = maybeMakeNegative(ZZ.random_element(1, 5))
#
evaluateFGQ1 = round(float(functionQ1f(x = functionQ1g(x = valueQ1))), 3)
evaluateGFQ1 = round(float(functionQ1g(x = functionQ1f(x = valueQ1))), 3)
###
###
degreeQ2f = ZZ.random_element(2, 5)
functionQ2f = generatePolynomial(degreeQ2f)
tempQ2 = generateRational()
functionQ2g = tempQ2[0]
valueQ2 = maybeMakeNegative(ZZ.random_element(1, 5))
while valueQ2 == tempQ2:
    valueQ2 = maybeMakeNegative(ZZ.random_element(2, 7))
#
evaluateFGQ2 = round(float(functionQ2f(x = functionQ2g(x = valueQ2))), 3)
evaluateGFQ2 = round(float(functionQ2g(x = functionQ2f(x = valueQ2))), 3)
###
###
degreeQ3f = ZZ.random_element(2, 5)
functionQ3f = generatePolynomial(degreeQ3f)
tempQ3 = generateRadical()
functionQ3g = tempQ3[0]
if tempQ3[1][0] > -Infinity:
    bound3 = tempQ3[1][0]
    valueQ3 = maybeMakeNegative(ZZ.random_element(1, 5))
    while valueQ3 < bound3 or functionQ3f(valueQ3) < bound3:
        degreeQ3f = ZZ.random_element(2, 5)
        functionQ3f = generatePolynomial(degreeQ3f)
        valueQ3 = maybeMakeNegative(ZZ.random_element(1, 5))
else:
    bound3 = tempQ3[1][1]
    valueQ3 = maybeMakeNegative(ZZ.random_element(1, 5))
    while valueQ3 > bound3 or functionQ3f(valueQ3) > bound3:
        degreeQ3f = ZZ.random_element(2, 5)
        functionQ3f = generatePolynomial(degreeQ3f)
        valueQ3 = maybeMakeNegative(ZZ.random_element(1, 5))
#
evaluateFGQ3 = round(float(functionQ3f(x = functionQ3g(x = valueQ3))), 3)
evaluateGFQ3 = round(float(functionQ3g(x = functionQ3f(x = valueQ3))), 3)
###
###
functionQ4f, functionQ4g, aQ4, bQ4 = generateInversePair()
valueQ4 = maybeMakeNegative(ZZ.random_element(2, 7))
#
evaluateFGQ4 = round(float(functionQ4f(x = evaluateNegativeCubeRoot(valueQ4, aQ4, bQ4) )), 3)
evaluateGFQ4 = round(float( evaluateNegativeCubeRoot(functionQ4f(x = valueQ4), aQ4, bQ4) ), 3)
###
 
 
 
\end{sagesilent}
 
% Q1 - Two polynomials
\begin{question}
For the two functions below, evaluate $(f\circ g)(\sage{valueQ1})$ and $(g \circ f)(\sage{valueQ1})$
 
$$f(x) = \sage{functionQ1f}$$
$$g(x) = \sage{functionQ1g}$$
 
$(f\circ g)(\sage{valueQ1}) = \answer[tolerance=0.05]{\sage{evaluateFGQ1}}$ \\
 
$(g \circ f)(\sage{valueQ1}) = \answer[tolerance=0.05]{\sage{evaluateGFQ1}}$
 
\begin{feedback}
Remember, the order is important! This joins subtraction and division where the order matters.
\end{feedback}
 
\end{question}
 
% Q2 - Polynomial and Rational
\begin{question}
For the two functions below, evaluate $(f\circ g)(\sage{valueQ2})$ and $(g \circ f)(\sage{valueQ2})$
 
$$f(x) = \sage{functionQ2f}$$
$$g(x) = \sage{functionQ2g}$$
 
$(f\circ g)(\sage{valueQ2}) = \answer[tolerance=0.05]{\sage{evaluateFGQ2}}$ \\
 
$(g \circ f)(\sage{valueQ2}) = \answer[tolerance=0.05]{\sage{evaluateGFQ2}}$
 
\begin{feedback}[correct]
Great! Beyond our first question, we needed to be careful that we could plug in our values, as $g(x)$ has a restricted domain.
\end{feedback}
 
\end{question}
 
% Q3 - Polynomial and Radical (not inverses)
\begin{question}
For the two functions below, evaluate $(f\circ g)(\sage{valueQ3})$ and $(g \circ f)(\sage{valueQ3})$
 
$$f(x) = \sage{functionQ3f}$$
$$g(x) = \sage{functionQ3g}$$
 
$(f\circ g)(\sage{valueQ3}) = \answer[tolerance=0.05]{\sage{evaluateFGQ3}}$ \\
 
$(g \circ f)(\sage{valueQ3}) = \answer[tolerance=0.05]{\sage{evaluateGFQ3}}$
 
\begin{feedback}[correct]
Great! Beyond our first question, we needed to be careful that we could plug in our values, as $g(x)$ has a restricted domain.
\end{feedback}
 
\end{question}
 
% Q4 - Polynomial and Radical (inverses)
\begin{question}
For the two functions below, evaluate $(f\circ g)(\sage{valueQ4})$ and $(g \circ f)(\sage{valueQ4})$
 
$$f(x) = \sage{functionQ4f}$$
$$g(x) = \sage{functionQ4g}$$
 
$(f\circ g)(\sage{valueQ4}) = \answer[tolerance=0.05]{\sage{evaluateFGQ4}}$ \\
 
$(g \circ f)(\sage{valueQ4}) = \answer[tolerance=0.05]{\sage{evaluateGFQ4}}$
\end{question}
 
%Q6 - Summary
\begin{question}
One of the biggest takeaways from this objective is noticing that $(f \circ g)(x) \neq (g \circ f)(x)$ \textbf{in most cases}.
 
For which question was $(f \circ g)(x) = (g \circ f)(x)$? $\answer{4}$
\begin{feedback}
It is just asking for the number of the question. The answer is either ``1", ``2", ``3", or ``4".
\end{feedback}
 
\begin{feedback}[correct]
Great job! But since it wasn't always the case, it should make you wonder: when is it that $(f \circ g)(x) = (g \circ f)(x)$? The next two objectives will answer this question.
\end{feedback}
\end{question}
 
\end{document}