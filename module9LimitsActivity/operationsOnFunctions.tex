\documentclass{ximera}
\usepackage{sagetex}
%% handout
%% space
%% newpage
%% numbers
%% nooutcomes
 
%% You can put user macros here
%% However, you cannot make new environments

\graphicspath{{./}{module1Activity/}{module2Activity/}{module3Activity/}}

\usepackage{sagetex}
\usepackage{tikz}
\usepackage{hyperref}
\usepackage{tkz-euclide}
\usetkzobj{all}
\pgfplotsset{compat=1.7} % prevents compile error.

\tikzstyle geometryDiagrams=[ultra thick,color=blue!50!black]
 %% we can turn off input when making a master document
 
\outcome{}
\author{Darryl Chamberlain Jr.}
  
\title{Objective 1 - Domain}
 
\begin{document}
\begin{abstract}
Identify the domain after operating $(+, -, \text{x}, \div)$ on functions.
\end{abstract}
\maketitle
 
\href{https://cnx.org/contents/mwjClAV_@8.1:ik_Ed0Pa@12/Composition-of-Functions}{Link to section in online textbook.}
 
%%%%%%%%%%%%%%%%%%%%%
%%%  Objective 1  %%%
%%%%%%%%%%%%%%%%%%%%%
 
First, watch
\underline{\href{https://mediasite.video.ufl.edu/Mediasite/Play/492b15b9c57c443d84b8f5bb37304a3d1d}{this video}} to see how operations change the domain of two functions. This is the first time we are treating \textbf{functions} like \textbf{numbers} that we can add/subtract/multiply/divide.
 
% Types of functions:
    % Polynomial - All Real numbers
    % Rational -  Square and Cube
    % Radical - Remove denominator
 
\begin{sagesilent}
x = var("x")
 
def maybeMakeNegative(natural):
    integer = natural*(-1)**ZZ.random_element(2)
    return integer
 
def generatePolynomial(degree):
    coefficients = [maybeMakeNegative(ZZ.random_element(3, 7)), maybeMakeNegative(ZZ.random_element(3, 7)), maybeMakeNegative(ZZ.random_element(3, 7)), maybeMakeNegative(ZZ.random_element(3, 7)), maybeMakeNegative(ZZ.random_element(3, 7))]
    if degree == 1:
        coefficients[4] = 0
        coefficients[3] = 0
        coefficients[2] = 0
    elif degree == 2:
        coefficients[4] = 0
        coefficients[3] = 0
    else:
        coefficients[4] = 0
    polynomial = coefficients[4] * x**4 + coefficients[3] * x**3 + coefficients[2] * x**2 + coefficients[1] * x + coefficients[0]
    return polynomial
 
def generateQuadratic():
    a1 = maybeMakeNegative(ZZ.random_element(1, 6))
    b1 = maybeMakeNegative(ZZ.random_element(1, 6))
    a2 = maybeMakeNegative(ZZ.random_element(1, 6))
    b2 = maybeMakeNegative(ZZ.random_element(1, 6))
    if -b1/a1 == -b2/a2 or a1 == 3 or a2 == 3:
        a1 = maybeMakeNegative(ZZ.random_element(1, 6))
        b1 = maybeMakeNegative(ZZ.random_element(1, 6))
        a2 = maybeMakeNegative(ZZ.random_element(1, 6))
        b2 = maybeMakeNegative(ZZ.random_element(1, 6))
    factor1 = a1*x + b1
    factor2 = a2*x + b2
    quadratic = factor1 * factor2
    if -b1/a1 < -b2/a2:
        return [quadratic, -b1/a1, -b2/a2]
    else:
        return [quadratic, -b2/a2, -b1/a1]
 
def generateRadical():
    factor = generatePolynomial(1)
    pivotPoint = round(float(solve(factor == 0, x)[0].rhs() ), 3)
    checkNearby = pivotPoint - 1
    if factor(x=checkNearby) < 0:
        domain = [pivotPoint, Infinity]
    else:
        domain = [-Infinity, pivotPoint]
    radical = sqrt(factor)
    return [radical, domain]
 
def generateRational():
    factor = generatePolynomial(1)
    excludeFromDomain = round(float(solve(factor == 0, x)[0].rhs() ), 3)
    rational = 1/factor
    return [rational, excludeFromDomain]
######
d1 = ZZ.random_element(1, 5)
f1 = generatePolynomial(d1)
g1 = generateQuadratic()
###
d2 = ZZ.random_element(1, 5)
f2 = generatePolynomial(d2)
g2 = generateRadical()
while g2[1][0] == -Infinity:
    g2 = generateRadical()
###
d3 = ZZ.random_element(1, 5)
f3 = generatePolynomial(d3)
g3 = generateRational()
###
g4 = generateRadical()
while g4[1][1] == Infinity:
    g4 = generateRadical()
f4 = generateRational()
while f4[1] > g4[1][1]:
    f4 = generateRational()
\end{sagesilent}
 
% Q1 - 2 poly, g(x) is a quadratic
 
\begin{question}
 
First, determine the domain of each function below. Then, determine the resulting domain after operating ($+, -, \text{x}, \div$) on the two functions. 
 
$$ f(x) = \sage{f1} $$
$$ g(x) = \sage{g1[0]} $$
 
Domain of $f(x)$: $\answer[format=string]{(} \answer{\sage{-Infinity}}, \answer{\sage{Infinity}} \answer[format=string]{)}$

Sample response: $(-\infty, \infty)$
 
\begin{feedback}
\textit{Note: There are boxes for you to put either "(" or "[" based on whether you want to include the end points. This is why there are four boxes.}
\end{feedback}
 
\begin{hint}
You are looking for the interval that describes the domain. If we do not need to remove numbers from the domain, the answer is $(-\infty, \infty)$. Otherwise, we need to restrict the domain like $(3, \infty)$.
\end{hint}
 
Domain of $g(x)$: $\answer[format=string]{(} \answer{\sage{-Infinity}}, \answer{\sage{Infinity}} \answer[format=string]{)}$  \\
 
Domain of $f(x) + g(x)$: $\answer[format=string]{(} \answer{\sage{-Infinity}}, \answer{\sage{Infinity}} \answer[format=string]{)}$  \\
 
Domain of $f(x) - g(x)$: $\answer[format=string]{(} \answer{\sage{-Infinity}}, \answer{\sage{Infinity}} \answer[format=string]{)}$  \\
 
Domain of $f(x) * g(x)$: $\answer[format=string]{(} \answer{\sage{-Infinity}}, \answer{\sage{Infinity}} \answer[format=string]{)}$  \\
 
Domain of $f(x) \div g(x)$:
$$\answer[format=string]{(} \answer{\sage{-Infinity}}, \answer{\sage{g1[1]}} \answer[format=string]{)}$$
$$\cup$$
$$\answer[format=string]{(} \answer[tolerance=0.05]{\sage{g1[1]}}, \answer[tolerance=0.05]{\sage{g1[2]}} \answer[format=string]{)}$$
$$\cup$$
$$\answer[format=string]{(} \answer[tolerance=0.05]{\sage{g1[2]}}, \answer{\sage{Infinity}} \answer[format=string]{)}$$
 
\begin{hint}
The denominator is already factored so you can focus on the new concept. How do we write the union of intervals when we have 2 points missing?
\end{hint}
 
\end{question}
 
% Q2 - Poly and Radical
 
\begin{question}
First, determine the domain of each function below. Then, determine the resulting domain after operating ($+, -, \text{x}, \div$) on the two functions.
 
$$ f(x) = \sage{f2} $$
$$ g(x) = \sage{g2[0]} $$
 
Domain of $f(x)$: $\answer[format=string]{(} \answer{\sage{-Infinity}}, \answer{\sage{Infinity}} \answer[format=string]{)}$
 
\begin{feedback}
\textit{Note: There are boxes for you to put either "(" or "[" based on whether you want to include the end points. This is why there are four boxes.}
\end{feedback}
 
\begin{hint}
You are looking for the interval that describes the domain. If we do not need to remove numbers from the domain, the answer is $(-\infty, \infty)$. Otherwise, we need to restrict the domain like $(3, \infty)$.
\end{hint}
 
Domain of $g(x)$: $\answer[format=string]{[} \answer[tolerance=0.05]{\sage{g2[1][0]}}, \answer{\sage{Infinity}} \answer[format=string]{)}$  \\
 
Domain of $f(x) + g(x)$: $\answer[format=string]{[} \answer[tolerance=0.05]{\sage{g2[1][0]}}, \answer{\sage{Infinity}} \answer[format=string]{)}$  \\
 
Domain of $f(x) - g(x)$: $\answer[format=string]{[} \answer[tolerance=0.05]{\sage{g2[1][0]}}, \answer{\sage{Infinity}} \answer[format=string]{)}$  \\
 
Domain of $f(x) * g(x)$: $\answer[format=string]{[} \answer[tolerance=0.05]{\sage{g2[1][0]}}, \answer{\sage{Infinity}} \answer[format=string]{)}$  \\
 
Domain of $f(x) \div g(x)$: $\answer[format=string]{(} \answer[tolerance=0.05]{\sage{g2[1][0]}}, \answer{\sage{Infinity}} \answer[format=string]{)}$
 
\begin{hint}
The domain for division is \textbf{nearly} the same. Do we need to exclude something extra since the radical is in the denominator?
\end{hint}
 
\end{question}
 
% Q3 - poly and rational
 
\begin{question}
 
First, determine the domain of each function below. Then, determine the resulting domain after operating ($+, -, \text{x}, \div$) on the two functions.
 
$$ f(x) = \sage{f3} $$
$$ g(x) = \sage{g3[0]} $$
 
Domain of $f(x)$: $\answer[format=string]{(} \answer{\sage{-Infinity}}, \answer{\sage{Infinity}} \answer[format=string]{)}$
 
\textit{Hint: You are looking for the interval that describes the domain. If we do not need to remove numbers from the domain, the answer is $(-\infty, \infty)$. Otherwise, we need to restrict the domain like $(3, \infty)$.} \\
 
Domain of $g(x)$:
$$\answer[format=string]{(} \answer{\sage{-Infinity}}, \answer[tolerance=0.05]{\sage{g3[1]}} \answer[format=string]{)}$$
$$\cup$$
$$\answer[format=string]{(} \answer[tolerance=0.05]{\sage{g3[1]}}, \answer{\sage{Infinity}} \answer[format=string]{)}$$  \\
 
Domain of $f(x) + g(x)$:
$$\answer[format=string]{(} \answer{\sage{-Infinity}}, \answer[tolerance=0.05]{\sage{g3[1]}} \answer[format=string]{)}$$
$$\cup$$
$$\answer[format=string]{(} \answer[tolerance=0.05]{\sage{g3[1]}}, \answer{\sage{Infinity}} \answer[format=string]{)}$$  \\
 
Domain of $f(x) - g(x)$:
$$\answer[format=string]{(} \answer{\sage{-Infinity}}, \answer[tolerance=0.05]{\sage{g3[1]}} \answer[format=string]{)}$$
$$\cup$$
$$\answer[format=string]{(} \answer[tolerance=0.05]{\sage{g3[1]}}, \answer{\sage{Infinity}} \answer[format=string]{)}$$  \\
 
Domain of $f(x) * g(x)$:
$$\answer[format=string]{(} \answer{\sage{-Infinity}}, \answer[tolerance=0.05]{\sage{g3[1]}} \answer[format=string]{)}$$
$$\cup$$
$$\answer[format=string]{(} \answer{\sage{g3[1]}}, \answer{\sage{Infinity}} \answer[format=string]{)}$$  \\
 
Domain of $f(x) \div g(x)$:
$$\answer[format=string]{(} \answer{\sage{-Infinity}}, \answer[tolerance=0.05]{\sage{g3[1]}} \answer[format=string]{)}$$
$$\cup$$
$$\answer[format=string]{(} \answer[tolerance=0.05]{\sage{g3[1]}}, \answer{\sage{Infinity}} \answer[format=string]{)}$$ 
 
\begin{hint}
You may be tempted to say that dividing by a rational will ``fix" the domain. When looking at the entire function $f(x) \div g(x)$, we want to make sure our values are defined on each function separately AND the resulting function.
\end{hint}
 
\end{question}
 
% Q4 - radical and rational
 
\begin{question}
 
First, determine the domain of each function below. Then, determine the resulting domain after operating ($+, -, \text{x}, \div$) on the two functions.
 
$$ f(x) = \sage{f4[0]} $$
$$ g(x) = \sage{g4[0]} $$
 
Domain of $f(x)$:
$$\answer[format=string]{(} \answer{\sage{-Infinity}}, \answer[tolerance=0.05]{\sage{f4[1]}} \answer[format=string]{)}$$
$$\cup$$
$$\answer[format=string]{(} \answer[tolerance=0.05]{\sage{f4[1]}}, \answer{\sage{Infinity}} \answer[format=string]{)}$$
\textit{Hint: You are looking for the interval that describes the domain. If we do not need to remove numbers from the domain, the answer is $(-\infty, \infty)$. Otherwise, we need to restrict the domain like $(3, \infty)$.} \\
 
Domain of $g(x)$:
$$\answer[format=string]{(} \answer{\sage{-Infinity}}, \answer[tolerance=0.05]{\sage{g4[1][1]}} \answer[format=string]{]}$$
 
Domain of $f(x) + g(x)$:
$$\answer[format=string]{(} \answer{\sage{-Infinity}}, \answer[tolerance=0.05]{\sage{f4[1]}} \answer[format=string]{)}$$
$$\cup$$
$$\answer[format=string]{(} \answer[tolerance=0.05]{\sage{f4[1]}}, \answer[tolerance=0.05]{\sage{g4[1][1]}} \answer[format=string]{]}$$  \\
 
Domain of $f(x) - g(x)$:
$$\answer[format=string]{(} \answer{\sage{-Infinity}}, \answer[tolerance=0.05]{\sage{f4[1]}} \answer[format=string]{)}$$
$$\cup$$
$$\answer[format=string]{(} \answer[tolerance=0.05]{\sage{f4[1]}}, \answer[tolerance=0.05]{\sage{g4[1][1]}} \answer[format=string]{]}$$  \\
 
Domain of $f(x) * g(x)$:
$$\answer[format=string]{(} \answer{\sage{-Infinity}}, \answer[tolerance=0.05]{\sage{f4[1]}} \answer[format=string]{)}$$
$$\cup$$
$$\answer[format=string]{(} \answer[tolerance=0.05]{\sage{f4[1]}}, \answer[tolerance=0.05]{\sage{g4[1][1]}} \answer[format=string]{]}$$  \\
 
Domain of $f(x) \div g(x)$:
$$\answer[format=string]{(} \answer{\sage{-Infinity}}, \answer[tolerance=0.05]{\sage{f4[1]}} \answer[format=string]{)}$$
$$\cup$$
$$\answer[format=string]{(} \answer[tolerance=0.05]{\sage{f4[1]}}, \answer[tolerance=0.05]{\sage{g4[1][1]}} \answer[format=string]{)}$$   
 
\end{question}
 
\begin{problem}
Look back on the last 4 questions and see if you can spot a pattern. This pattern is our main takeaway for this objective: \\ \\
 
The resulting domain is the $\answer[format=string]{intersection}$ of the two parent domains. If we are looking at the special case of dividing, we also need to remove $g(x) = \answer{0}$ from the $\answer[format=string]{intersection}$ of the two parent domains.
 
\begin{feedback}[correct]
In Calculus, you will learn new operations on functions. One of the most important things to keep in mind is the relationship between the domain of the ``parent function(s)" and the resulting function.
\end{feedback}
 
\end{problem}
 
\end{document}