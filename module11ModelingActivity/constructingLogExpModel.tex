\documentclass{ximera}
\usepackage{sagetex}
%% handout
%% space
%% newpage
%% numbers
%% nooutcomes
 
%% You can put user macros here
%% However, you cannot make new environments

\graphicspath{{./}{module1Activity/}{module2Activity/}{module3Activity/}}

\usepackage{sagetex}
\usepackage{tikz}
\usepackage{hyperref}
\usepackage{tkz-euclide}
\usetkzobj{all}
\pgfplotsset{compat=1.7} % prevents compile error.

\tikzstyle geometryDiagrams=[ultra thick,color=blue!50!black]
 %% we can turn off input when making a master document
 
\outcome{}
\author{Darryl Chamberlain Jr.}
  
\title{Objective 3 - Construct Log/Exp Models}
 
\begin{document}
\begin{abstract}

\end{abstract}

\maketitle
 
% Link to textbook
Link to textbook: 
\href{https://cnx.org/contents/mwjClAV_@8.21:_tqWoaDz@17/Exponential-and-Logarithmic-Models}{Construct a model equation for the real-life situation.} 

You can print out \href{http://people.clas.ufl.edu/dchamberlain31/files/M11M-Objective-3-Construct-Log-Exp-Model.pdf}{these notes} to follow along with the video below and keep notes to organize your thoughts.

\youtube{QUL7SUcfXJE}
 
%%%%%%%%%%%%%%%%%%%%%
%%%  Objective 3  %%%
%%%%%%%%%%%%%%%%%%%%%

%DOUBLING/TRIPILING/QUADRUPLING TIME
\begin{sagesilent}
t = var('t')
rateList1 = ["doubles", "triples", "quadruples"]
choose1 = ZZ.random_element(0, 3)
rate1 = rateList1[choose1]
initialBacteria1 = ZZ.random_element(1, 9)*100
k1 = ln(choose1+2)
bacteriaPopulation1 = initialBacteria1 * (choose1 + 2)**(t)
\end{sagesilent}
\begin{question}
A population of bacteria $\sage{rate1}$ every hours. If the culture started with $\sage{initialBacteria1}$, write the equation that models the bacteria population after $t$ hours. 

$P(t) = \answer{\sage{initialBacteria1}} \answer{\sage{choose1+2}}^{\answer{t}}$

\end{question}

% HALF-LIFE
\begin{sagesilent}
t = var('t')
halfLifeYears2 = ZZ.random_element(100, 1001)*ZZ.random_element(100, 1001)
initialAmount2 = ZZ.random_element(100, 1000)
k2 = -ln(2)/halfLifeYears2
equation2 = initialAmount2*e**(k2*t)
\end{sagesilent}
\begin{question}
There is initially $\sage{initialAmount2}$ grams of element $X$. The half-life of element $X$ is $\sage{halfLifeYears2}$ years. Describe the amount of element $X$ remaining as a function of time, $t$, in years. Use $e$ as the base of your exponential model and exact values.

$X(t) = \answer{\sage{initialAmount2}} e^{\answer{\sage{k2}t}  }$

\begin{hint}
	The exponent's coefficient is not $\frac{1}{\sage{halfLifeYears2}}$ nor is it $\frac{-1}{\sage{halfLifeYears2}}$. To find the correct coefficient, take the base exponential equation $\text{current amount } = \text{ initial amount } e^{ct}$,  use what you know about the half-life time and amount, then solve the equation for $c$. If you are having trouble, refer back to the objective on "Solving exponential equations". Do not use a calculator - use exact values to get a correct answer. 
\end{hint}
\end{question}

% Carbon Dating
\begin{sagesilent}
t = var('t')
r = var('r')
k3 = ln(0.5)/5730
equation3A = e**(t*k3)
equation3B = ln(r)/k3
percent3 = ZZ.random_element(1, 101)
old3 = round(equation3B(percent3*0.01), 0)
\end{sagesilent}

\begin{question}
The half-life of carbon-14 is 5,730 years. 

\textbf{Part A.} Describe the amount of carbon-14 remaining after $t$ years. The initial amount of carbon-14, $C_0$, is already included below. 

$C = C_0 * \answer{e}^{\answer{\sage{k3}t} }$

\textbf{Part B.} Solve the equation above for $t$ written in terms of the ratio of carbon-14 remaining, $r = \frac{C}{C_0}$.

$t = \answer[tolerance=0.001]{\sage{1/k3}} \ln(\answer{r}) $

\textbf{Part C.} The equation above is used to carbon-date objects. \textit{To solidify this idea, use the model in Part B. to solve the following problem.}

A bone fragment is found that contains $\sage{percent3}\%$ of its original carbon-14. To the nearest year, how old is the bone?

$\answer{\sage{old3}} \, \text{ years old}$

\begin{hint}
	\textbf{Part A.} The exponent's coefficient is not $\frac{1}{5730}$ nor is it $\frac{-1}{5730}$. To find the correct coefficient, take the base exponential equation $\text{current amount } = \text{ initial amount } e^{ct}$,  use what you know about the half-life time and amount, then solve the equation for $c$. If you are having trouble, refer back to the objective on "Solving exponential equations". Do not use a calculator - use exact values to get a correct answer. 
	
	\textbf{Part B.} Now that you found the exponent's coefficient, you have modeled the general equation as $C = C_0 e^{ct}$. Since it asks you to rewrite the equation in terms of $r = \frac{C}{C_0}$, start by converting the equation into one with $r$ instead of $C$ and $C_0$. Then, solve the equation for $t$.
	
	\textbf{Part C.} What does $r = \frac{C}{C_0}$ represent? If we know this, then we could use the equation you built in Part B. to solve for the age of the bone. 
\end{hint}

\end{question}

\end{document}