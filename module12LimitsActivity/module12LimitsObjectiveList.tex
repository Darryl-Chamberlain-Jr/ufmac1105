\documentclass{ximera}
%\usepackage{sagetex}
%% handout
%% space
%% newpage
%% numbers
%% nooutcomes

%%% You can put user macros here
%% However, you cannot make new environments

\graphicspath{{./}{module1Activity/}{module2Activity/}{module3Activity/}}

\usepackage{sagetex}
\usepackage{tikz}
\usepackage{hyperref}
\usepackage{tkz-euclide}
\usetkzobj{all}
\pgfplotsset{compat=1.7} % prevents compile error.

\tikzstyle geometryDiagrams=[ultra thick,color=blue!50!black]
 %% we can turn off input when making a master document

\outcome{}
\author{Darryl Chamberlain Jr.}
 
\title{Objectives}

\begin{document}
\begin{abstract}
List of objectives for Module 12L - Graphing Rational Functions. 
\end{abstract}

\maketitle

% Introduction to the section. 
In this last Module, we go back to rational functions. Previously, we only graphed ``basic" rational functions that were in the form

$$ f(x) = \frac{a}{(x-h)^n} + k $$

Now that we have covered how to divide two polynomials (Synthetic Division) and explored the limit of a function as it approaches a value, we can graph ``general" rational functions of the form 

$$ f(x) = \frac{a_n x^n + a_{n-1} x^{n-1} + \cdots + a_1 x + a_0}{b_n x^n + b_{n-1} x^{n-1} + \cdots + b_1 x + b_0}$$

% How does it relate to previous section?

% How will it be useful for the future?


The objectives for this homework are: 
\begin{enumerate}
\item \href{https://cnx.org/contents/mwjClAV_@8.21:KNTP2r7D@14/Rational-Functions}{Use limits to determine the holes of a rational function.}

\item \href{https://cnx.org/contents/mwjClAV_@8.21:KNTP2r7D@14/Rational-Functions}{Use limits to determine the vertical asymptotes of a rational function.}

\item \href{https://cnx.org/contents/mwjClAV_@8.21:KNTP2r7D@14/Rational-Functions}{Use limits to determine the horizontal asymptotes of a rational function.}

\item \href{https://cnx.org/contents/mwjClAV_@8.21:KNTP2r7D@14/Rational-Functions}{Use limits to determine the oblique asymptotes of a rational function.}
\end{enumerate}

\end{document}