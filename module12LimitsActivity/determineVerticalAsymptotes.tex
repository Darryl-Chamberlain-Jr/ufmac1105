\documentclass{ximera}
\usepackage{sagetex}
%% handout
%% space
%% newpage
%% numbers
%% nooutcomes
 
%% You can put user macros here
%% However, you cannot make new environments

\graphicspath{{./}{module1Activity/}{module2Activity/}{module3Activity/}}

\usepackage{sagetex}
\usepackage{tikz}
\usepackage{hyperref}
\usepackage{tkz-euclide}
\usetkzobj{all}
\pgfplotsset{compat=1.7} % prevents compile error.

\tikzstyle geometryDiagrams=[ultra thick,color=blue!50!black]
 %% we can turn off input when making a master document
 
\outcome{}
\author{Darryl Chamberlain Jr.}
  
\title{Objective 2 - Vertical Asymptotes}
 
\begin{document}
\begin{abstract}

\end{abstract}
\maketitle
 
\href{https://cnx.org/contents/mwjClAV_@8.21:KNTP2r7D@14/Rational-Functions}{Link to section in online textbook.}

\href{https://mediasite.video.ufl.edu/Mediasite/Play/7c895945aa40406e93f9a39e2aa343561d}{Introduction video describing holes/vertical asymptotes \textit{without limits}.}
 
%%%%%%%%%%%%%%%%%%%%%
%%%  Objective 2  %%%
%%%%%%%%%%%%%%%%%%%%%

When we learned about the domain of rational functions, we set the denominator equal to 0 and solved. This gave us all values that the function is not defined for. As we saw in Objective 1, some of these $x$-values are holes of the function. The rest are \textbf{vertical asymptotes} of the function, or vertical lines where the function approaches positive or negative infinity. 

\[
\graph[rectangular]{f(x) = 1/((x-1)(x-3))}
\]

Unlike holes, vertical asymptotes affect the function around where they are defined. With left- and right-sided limits, we can determine how the function is behaving near these vertical lines. 

\begin{theorem}
\textbf{Vertical Asymptotes of a Rational Function}

A rational function has a vertical asymptote $x = a$ (vertical line) if 

$$\lim_{x \rightarrow a^{-}} f(x) = \pm \infty \text{ or } \lim_{x \rightarrow a^{+}} f(x) = \pm \infty $$ 

Thus, to determine if a rational function has any vertical asymptotes, we need to factor the denominator and evaluate the limit. If a one-sided limit is positive or negative infinity, it is a vertical asymptote. 
\end{theorem}

Practice with the questions below. 

\begin{sagesilent}
R, x = QQ['x'].objgen()
 
def maybeMakeNegative(natural):
    integer = natural*(-1)**ZZ.random_element(2)
    return integer
 
def makeIntegerFactor():
    zero = maybeMakeNegative(ZZ.random_element(1, 6))
    integerFactor = x - zero
    return [integerFactor, zero]
 
def makeRationalFactor():
    a = maybeMakeNegative(ZZ.random_element(1, 4))
    b = maybeMakeNegative(ZZ.random_element(1, 6))
    while gcd(abs(a), abs(b)) > 1:
        a = maybeMakeNegative(ZZ.random_element(1, 4))
        b = maybeMakeNegative(ZZ.random_element(1, 6))
    rationalFactor = a*x + b
    return [rationalFactor, -b/a]
 
def makeIrrationalQuadratic():
    a = maybeMakeNegative(ZZ.random_element(1, 6))
    b = maybeMakeNegative(ZZ.random_element(1, 6))
    c = maybeMakeNegative(ZZ.random_element(1, 6))
    discrim = b**2 - 4*a*c
    integerType = type(2)
    while type(sqrt(discrim)) == integerType:
        a = maybeMakeNegative(ZZ.random_element(1, 6))
        b = maybeMakeNegative(ZZ.random_element(1, 6))
        c = maybeMakeNegative(ZZ.random_element(1, 6))
        discrim = b**2 - 4*a*c
    solution0 = (-b + sqrt(discrim))/2*a
    solution1 = (-b - sqrt(discrim))/2*a
    smallerSolution, largerSolution = sorted([solution0, solution1])
    quadratic = a*x**2 + b*x + c
    return [quadratic, smallerSolution, largerSolution]
 
def makeComplexQuadratic():
    a0 = maybeMakeNegative(ZZ.random_element(1, 4))
    b0 = maybeMakeNegative(ZZ.random_element(1, 6))
    complex0 = a0 + b0*i
    complex1 = a0 - b0*i
    quadratic = x**2 + (a0**2 + b0**2)
    return [quadratic, complex0, complex1]
##########
function1num, answer1 = makeIntegerFactor()
function1denom = ZZ.random_element(2, 7) * function1num
###
function2num = maybeMakeNegative(ZZ.random_element(2, 7))
function2denom, answer2 = makeIntegerFactor()
###
function3num = maybeMakeNegative(ZZ.random_element(2, 7))
factor3denomA, answer3aTEMP = makeRationalFactor()
factor3denomB, answer3bTEMP = makeRationalFactor()
while answer3aTEMP == answer3bTEMP:
    factor3denomA, answer3aTEMP = makeRationalFactor()
    factor3denomB, answer3bTEMP = makeRationalFactor()
function3denom = factor3denomA * factor3denomB
answer3a, answer3b = sorted([answer3aTEMP, answer3bTEMP])
###
factor4numA, deadValue4A = makeRationalFactor()
factor4numB, deadValue4B = makeRationalFactor()
factor4denomA, answer4aTEMP = makeRationalFactor()
factor4denomB, answer4bTEMP = makeRationalFactor()
factor4denomC, answer4cTEMP = makeIntegerFactor()
while answer4aTEMP == answer4bTEMP or answer4aTEMP == answer4bTEMP or answer4bTEMP == answer4cTEMP or deadValue4A == answer4aTEMP or deadValue4A == answer4bTEMP or deadValue4A == answer4cTEMP or deadValue4B == answer4aTEMP or deadValue4B == answer4bTEMP or deadValue4B == answer4cTEMP or deadValue4A == deadValue4B:
    factor4numA, deadValue4A = makeRationalFactor()
    factor4numB, deadValue4B = makeRationalFactor()
    factor4denomA, answer4aTEMP = makeRationalFactor()
    factor4denomB, answer4bTEMP = makeRationalFactor()
    factor4denomC, answer4cTEMP = makeIntegerFactor()
function4num = factor4numA * factor4numB
function4denom = factor4denomA * factor4denomB * factor4denomC
answer4a, answer4b, answer4c = sorted([answer4aTEMP, answer4bTEMP, answer4cTEMP])
###
factor5numA, deadValue5A = makeRationalFactor()
factor5numB, deadValue5B = makeRationalFactor()
factor5denomA = factor5numA
answer5aTEMP = deadValue5A
factor5denomB, answer5bTEMP = makeRationalFactor()
factor5denomC, answer5cTEMP = makeIntegerFactor()
while answer5aTEMP == answer5bTEMP or answer5aTEMP == answer5bTEMP or answer5bTEMP == answer5cTEMP or deadValue5A == answer5bTEMP or deadValue5A == answer5cTEMP or deadValue5B == answer5aTEMP or deadValue5B == answer5bTEMP or deadValue5B == answer5cTEMP or deadValue5A == deadValue5B:
    factor5numA, deadValue5A = makeRationalFactor()
    factor5numB, deadValue5B = makeRationalFactor()
    factor5denomA = factor5numA
    answer5aTEMP = deadValue5A
    factor5denomB, answer5bTEMP = makeRationalFactor()
    factor5denomC, answer5cTEMP = makeIntegerFactor()
function5num = factor5numA * factor5numB
function5denom = factor5denomA * factor5denomB * factor5denomC
answer5a = answer5aTEMP
answer5b, answer5c = sorted([answer5bTEMP, answer5cTEMP])
###
factor6numA, deadValue6A = makeIntegerFactor()
factor6numB, deadValue6B = makeIntegerFactor()
factor6denomA = factor6numA
answer6 = deadValue6A
complexQuadratic6, deadComplex6A, deadComplex6B = makeComplexQuadratic()
function6num = factor6numA * factor6numB
function6denom = factor6denomA * complexQuadratic6
\end{sagesilent}
 
% Q1 - No VA, f(x) = (x-a)/(x-a)
\begin{question}
Find all vertical asymptotes of the rational function below. If they do not exist, answer ``NA".
 
$$ f(x) = \frac{\sage{function1num}}{\sage{function1denom}} $$
 
Vertical Asymptote: \textit{the vertical line} $x =  \answer[format=string]{NA}$.
 
\end{question}
 
% Q2 - Basic, f(x) = 1/(x-a)
\begin{question}
Find all vertical asymptotes of the rational function below. If they do not exist, answer ``NA".
 
$$ f(x) = \frac{\sage{function2num}}{\sage{function2denom}} $$
 
Vertical Asymptote: \textit{the vertical line} $x =  \answer[tolerance=0.05]{\sage{answer2}}$.
 
\end{question}
 
% Q3 - Need to factor, no holes, f(x) = 1/(a_0 x - a_1)(b_0 x - b_1)
\begin{question}
Find all vertical asymptotes of the rational function below. If they do not exist, answer ``NA".
 
$$ f(x) = \frac{\sage{function3num}}{\sage{function3denom}} $$
 
Vertical Asymptote: \textit{the vertical line} $x =  \answer[tolerance=0.05]{\sage{answer3a}}$ and $x = \answer[tolerance=0.05]{\sage{answer3b}}$.
 
% Remember to give tolerances when necessary
\end{question}
 
%Q4 - Need to factor cubic denominator, no holes, f(x) = (d_0 x - d_1)(e_0 x - e_1)/(x-a)(b_0 x - b_1)(c_0 x - c_1)
\begin{question}
Find all vertical asymptotes of the rational function below. If they do not exist, answer ``NA".
 
$$ f(x) = \frac{\sage{function4num}}{\sage{function4denom}} $$
 
Vertical Asymptote: \textit{the vertical line} $x =  \answer[tolerance=0.05]{\sage{answer4a}}$, $x =  \answer[tolerance=0.05]{\sage{answer4b}}$, and $x =  \answer[tolerance=0.05]{\sage{answer4c}}$.
 
\end{question}
 
%Q5 - Need to factor cubic denominator, 1 holes, f(x) = (d_0 x - d_1)(b_0 x - b_1)/(x-a)(b_0 x - b_1)(c_0 x - c_1)
\begin{question}
Find all vertical asymptotes of the rational function below. If they do not exist, answer ``NA".
 
$$ f(x) = \frac{\sage{function5num}}{\sage{function5denom}} $$
 
Vertical Asymptote: \textit{the vertical line} $x =  \answer[tolerance=0.05]{\sage{answer5b}}$ and $x = \answer[tolerance=0.05]{\sage{answer5c}}$.
 
\end{question}
 
%Q6 - Need to factor cubic, complex denominator, f(x) = (x-a)(x-b)/(x-a)(x-bi)(x+bi)
\begin{question}
Find all vertical asymptotes of the rational function below. If they do not exist, answer ``NA".
 
$$ f(x) = \frac{\sage{function6num}}{\sage{function6denom}} $$
 
Vertical Asymptote: \textit{the vertical line} $x =  \answer[format=string]{NA}$.
 
\end{question}
 
 
\textbf{Main takeaway:} Values not in the domain of the function can be one of two things:
\begin{itemize}
\item Vertical Asymptotes: Values the function will come close to, but will not touch at. These are vertical lines $x = a$, where $a$ makes the denominator zero. The limit of the function at these points are $\pm \infty$.
\item Holes: Values that only affect the function exactly at that point (rather than nearby by vertical asymptotes). The limit of the function at these points are $\frac{0}{0}$.
\end{itemize}

\end{document}